
% Copyright 2022 by Robert Hildebrand
%This work is licensed under a
%Creative Commons Attribution-ShareAlike 4.0 International License (CC BY-SA 4.0)
%See http://creativecommons.org/licenses/by-sa/4.0/

\todoChapter{ {\color{gray}0\% complete. Goal 80\% completion date: January 20, 2023}\\
Notes: Need to write this section.  Add exmaples from lecture notes.  Create code to help generate examples.}

Sensitivity analysis is an important tool for understanding the behavior of linear programs. It allows us to analyze how the optimal solution of a linear program changes as the coefficients of the constraints or the objective function are varied within certain bounds. In this section, we will present the basic concepts of sensitivity analysis and provide two examples to illustrate its use.

The first step in sensitivity analysis is to solve the linear program using a method such as the simplex method. Once the optimal solution has been obtained, we can then analyze how the solution changes as the coefficients of the constraints or the objective function are varied. For example, consider the following linear program:

\begin{align*}
\text{Maximize} \quad & 3x_1 + 2x_2 \
\text{Subject to} \quad & x_1 + x_2 \leq 4 \
& 2x_1 + x_2 \leq 5 \
& x_1, x_2 \geq 0
\end{align*}

The optimal solution to this linear program is $x_1 = 2$, $x_2 = 2$, with an optimal objective value of $3x_1 + 2x_2 = 10$.

To perform sensitivity analysis, we can consider how the optimal solution changes as the right-hand side (RHS) of the constraints is varied. For example, suppose we increase the RHS of the first constraint by 1, to 5. This corresponds to the modified constraint $x_1 + x_2 \leq 5$. The optimal solution to this modified linear program is $x_1 = 2$, $x_2 = 3$, with an optimal objective value of $3x_1 + 2x_2 = 12$. Thus, we can see that increasing the RHS of the first constraint by 1 has led to an increase in the optimal objective value.

Another example of sensitivity analysis is consider the effect of changing the coefficient of the objective function. For example, suppose we multiply the coefficient of $x_1$ by 2, to obtain the modified objective function $6x_1 + 2x_2$. The optimal solution to this modified linear program is $x_1 = 1$, $x_2 = 2$, with an optimal objective value of $6x_1 + 2x_2 = 8$. Thus, we can see that multiplying the coefficient of $x_1$ by 2 has led to a decrease in the optimal objective value.

In summary, sensitivity analysis is a useful tool for understanding how the optimal solution of a linear program changes as the coefficients of the constraints or the objective function are varied. It allows us to determine how robust the solution is to changes in the input data, and to identify the most critical factors affecting the solution.