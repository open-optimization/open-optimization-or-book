% Copyright 2020 by Robert Hildebrand
%This work is licensed under a
%Creative Commons Attribution-ShareAlike 4.0 International License (CC BY-SA 4.0)
%See http://creativecommons.org/licenses/by-sa/4.0/

%\documentclass[../open-optimization/open-optimization.tex]{subfiles}
%
%\begin{document}

%\section{Reference guide for modeling and examples}
%\url{https://download.aimms.com/aimms/download/manuals/AIMMS3_OM.pdf}

\section{MIP Solvers and Modeling Tools}

\begin{itemize}
\item AMPL
\item GAMS
\item AIMMS
\item Python-MIP
\item Pyomo
\item PuLP
\item JuMP
\end{itemize}

\begin{itemize}
\item GUROBI
\item CPLEX (IBM)
\item Express
\item SAS
\item Coin-OR (CBC, CLP, IPOPT)
\item SCIP
\end{itemize}




\subsection{Tools for Solving Job Shop Scheduling Problems}

Job Shop Scheduling (JSS) is a classic optimization problem in operations research. There are several tools and techniques available to tackle this problem, ranging from exact algorithms to heuristic and metaheuristic methods.

\begin{enumerate}
    \item \textbf{Exact Algorithms}:
    \begin{itemize}
        \item \textbf{Branch and Bound}: This is a general algorithm for finding optimal solutions. It systematically searches for the best solution by exploring all possible solutions in a tree-like structure.
        \item \textbf{Integer Linear Programming (ILP)}: Many commercial solvers like IBM CPLEX, Gurobi, and SCIP can be used to model and solve JSS as an ILP problem.
    \end{itemize}

    \item \textbf{Heuristic Methods}:
    \begin{itemize}
        \item \textbf{Priority Dispatching Rules}: These are simple rules like Shortest Processing Time (SPT), Earliest Due Date (EDD), and Longest Processing Time (LPT) that prioritize jobs based on certain criteria.
        \item \textbf{Shifting Bottleneck}: This heuristic focuses on the most constrained machine (the bottleneck) and schedules jobs on it first.
    \end{itemize}

    \item \textbf{Metaheuristic Methods}:
    \begin{itemize}
        \item \textbf{Genetic Algorithms (GA)}: GA is inspired by the process of natural selection. Tools like JGAP (Java Genetic Algorithms Package) can be used to implement GA for JSS.
        \item \textbf{Simulated Annealing (SA)}: SA is inspired by the annealing process in metallurgy. It's a probabilistic technique used to find an approximate solution to an optimization problem.
        \item \textbf{Tabu Search}: This is a local search method that uses memory structures to avoid getting trapped in local optima.
        \item \textbf{Particle Swarm Optimization (PSO)}: Inspired by the social behavior of birds, PSO is used to find optimal or near-optimal solutions.
    \end{itemize}

    \item \textbf{Hybrid Methods}:
    \begin{itemize}
        \item Combining two or more of the above methods can often yield better results. For example, a GA can be combined with SA or Tabu Search to enhance the search process.
    \end{itemize}

    \item \textbf{Software and Libraries}:
    \begin{itemize}
        \item \textbf{OptaPlanner}: An open-source constraint satisfaction solver in Java that can handle JSS problems.
        \item \textbf{OR-Tools by Google}: Provides a suite of operations research tools, including solvers for routing, linear programming, and constraint programming, which can be applied to JSS.
    \end{itemize}

    \item \textbf{Simulation Software}:
    \begin{itemize}
        \item Tools like FlexSim, AnyLogic, and Arena can be used to simulate and optimize job shop scheduling scenarios.
    \end{itemize}

    \item \textbf{Commercial Packages}:
    \begin{itemize}
        \item Some ERP (Enterprise Resource Planning) and MES (Manufacturing Execution Systems) software offer built-in tools for job shop scheduling.
    \end{itemize}
\end{enumerate}

When choosing a tool or method, it's essential to consider the specific requirements of the problem, the size of the problem instance, and the available computational resources. For smaller instances, exact methods might be feasible, but for larger, real-world problems, heuristic or metaheuristic methods are often more appropriate.





%%

%\end{document}
