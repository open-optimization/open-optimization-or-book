% Copyright 2020 by Robert Hildebrand
%This work is licensed under a
%Creative Commons Attribution-ShareAlike 4.0 International License (CC BY-SA 4.0)
%See http://creativecommons.org/licenses/by-sa/4.0/% 



%\documentclass[../open-optimization/open-optimization.tex]{subfiles}
% 
%\begin{document}

\chapter{Integer Programming Formulations}
\todoChapter{ 70\% complete. Goal 80\% completion date: August 20\\
Notes: }
\label{sec:IP-formulations} 
\begin{outcome}
\begin{enumerate}
\item[A.] Learn classic integer programming formulations.
\item[B.] Demonstrate different uses of binary and integer variables.
\item[C.]  Demonstrate the format for modeling an optimization problem with sets, parameters, variables, and the model.
\end{enumerate}
\end{outcome}
%\begin{comment}
%We will learn about complexity theory in a subsequent section.  But take note of the complexity classification of each problem ( \polynomial,\npcomplete, \nphard) to get a sense of how difficult the problem is.  We will define these terms later.
%\end{comment}




In this section, we will describe classical integer programming formulations.     These formulations may reflect  a real world problem exactly, or may be part of the setup of a real world problem.   

\section{Knapsack Problem}
The \emph{knapsack problem} can take different forms depending on if the variables are binary or integer.  The binary version means that there is only one item of each item type that can be taken.  This is typically illustrated as a backpack (knapsack) and some items to put into it (see  \autoref{fig:wiki/File/knapsack}), but has applications in many contexts.  



 \includefigurestatic[Knapsack Problem: which items should we choose take in the knapsack that maximizes the value while respecting the 15kg weight limit?][width=.4\linewidth][h]{wiki/File/knapsack}

\begin{general}{Binary Knapsack Problem}{\npcomplete}
Given an non-negative weight vector $a \in \Q^n_+$, a capacity $b \in \Q_+$, and objective coefficients $c \in \Q^n$, 
\begin{equation}
\begin{split}
\max \ \ & c^\top x\\
\text{s.t.}\ \ & a^\top x \leq b\\
& x \in \{0,1\}^n
\end{split}
\end{equation}
\end{general}
 


 
%\begin{figure}[H]
%\begin{center}
%\includegraphics[scale = 0.2]{knapsack}\footnotemark 
%\end{center}
%\label{fig:knapsack}
%\caption{Knapsack Problem: which items should we choose take in the knapsack that maximizes the value while respecting the 15kg weight limit?}
%\end{figure}
%\footnotetext{\url{https://en.wikipedia.org/wiki/Knapsack_problem}}

\begin{examplewithcode}{Knapsack}{https://github.com/open-optimization/open-optimization-or-examples/blob/master/integer-programming/knapsack-problem.ipynb}
\label{example:knapsack}
You have a knapsack (bag) that can only hold W = 15 kgs.  There are 5 items that you could possibly put into your knapsack.  The items (weight, value) are given as:
(12 kg, $\$$4), (2 kg, $\$$2), (1kg, $\$$2), (1kg, $\$$1), (4kg, $\$$10).  Which items should you take to maximize your value in the knapsack? See \autoref{fig:wiki/File/knapsack}.\\

\noindent \textbf{Variables:}
\begin{itemize}
\item let $x_i = 0$ if item $i$ is in the bag
\item let $x_i = 1$ if item $i$ is not in the bag
\end{itemize}
\textbf{Model:}
\begin{align}
\max  \  \ &4 x_1 + 2 x_2 + 2 x_3 + 1 x_4 + 10 x_5 \tag{Total value}\\
\text{ s.t. }\ \ &  12 x_1 + 2 x_2 + 1 x_3 + 1 x_4 + 4 x_5 \leq 15 \tag{Capacity bound}\\
& x_i \in \{0,1\} \text{ for } i=1, \dots, 5 \tag{Item taken or not}
\end{align}
\end{examplewithcode}
In the integer case, we typically require the variables to be non-negative integers, hence we use the notation $x \in \Z^n_+$.  This setting reflects the fact that instead of single individual items, you have item types of which you can take as many of each type as you like that meets the constraint.
\begin{general}{Integer Knapsack Problem}{\npcomplete}
Given an non-negative weight vector $a \in \Q^n_+$, a capacity $b \in \Q_+$, and objective coefficients $c \in \Q^n$, 
\begin{equation}
\begin{split}
\max \ \ & c^\top x\\
\text{s.t.}\ \ & a^\top x \leq b\\
& x \in \Z^n_+
\end{split}
\end{equation}
\end{general}
We can also consider an equality constrained version
\begin{general}{Equality Constrained Integer Knapsack Problem}{\nphard}
Given an non-negative weight vector $a \in \Q^n_+$, a capacity $b \in \Q_+$, and objective coefficients $c \in \Q^n$, 
\begin{align}
\max \ \ & c^\top x\\
\text{s.t.}\ \ & a^\top x = b\\
& x \in \Z^n_+
\end{align}
\end{general}
\begin{example}
\label{ex:min-coins}
Using pennies, nickels, dimes, and quarters, how can you minimize the number of coins you need to to make up a sum of $83\cent$? 

\textbf{Variables:}
\begin{itemize}
\item Let $p$ be the number of pennies used
\item Let $n$ be the number of nickels used
\item Let $d$ be the number of dimes used
\item Let $q$ be the number of quarters used
\end{itemize}
\textbf{Model}
\begin{align*}
\min \quad & p + n + d + q & \text{ total number of coins used}\\
\text{ s.t. } \quad & p + 5n + 10d + 25 q = 83 & \text{sums to } 83 \cent\\
& p,d,n,q \in \Z_+ & \text{each is a non-negative integer}
\end{align*}
\end{example}
\section{Capital Budgeting}
\todoSection{}

% Copywrite Robert Hildeband 2019


	\textcolor{blue}{A firm has $n$ projects it could undertake to maximize revenue, but budget limitations require that not all can be completed.}\\
	\textcolor{blue}{Project $j$ expects to produce revenue $c_j$\\
		Project $j$ requires investment $a_{ij}$ in time period $i$ for $i = 1,\ldots,m$\\
		In time period $i$, capital $b_i$ is available}
		
		Let $x_i$ be a binary variable such that $x_i = 1$ if we choose investment $i$ and $x_i = 0$ otherwise.  The the model can be given as:
	\begin{align*}
	\max ~~~& \sum_{j = 1}^n c_jx_j\\
	s.t. ~~~&\sum_{j = 1}^n a_{ij}x_j\leq b_i, ~~~i = 1,\ldots m\\
	& x_j \in \{0,1\} , \ j = 1,\ldots,n
	\end{align*}


Consider the example given in the following table.
	\begin{table}[h]
		\centering
		\resizebox{\columnwidth}{!}{%
			\begin{tabular}{|c|c|c|c|}%[<+->]
				\hline
				Project & $\mathbb{E}$[Revenue] & Resources required in week 1 & Resources required in week 2\\\hline
				\rowcolor{bluegreen} 1 & 10 & 3 & 4\\
				\hline
				2 & 8 & 1 & 2\\\hline
				\rowcolor{bluegreen} 3 & 6 & 2 & 1\\
				\hline
				Resources available & & 5 & 6\\
				\hline
		\end{tabular}}
	\end{table}
	$$
	\max~~~~~ 10x_1+8x_2+6x_3
	$$
	subject to
	\begin{align*}
	3x_1+1x_2+2x_3\leq5\\
	4x_1+2x_2+1x_3\leq6\\
	x_j \in \{0,1\},\  j = 1,2,3
	\end{align*}



\section{Set Covering}
\todoSection{}

The \emph{set covering} problem can be used for a wide array of problems.    We will see several examples in this section.

\begin{general}{Set Covering}{\npcomplete}
\label{general:set-covering}
Given a set $V$ with subsets $V_1, \dots, V_l$, determine the smallest subset $S \subseteq V$ such that 
$S \cap V_i \neq \emptyset$ for all $i=1, \dots, l$.

The set cover problem can be modeled as
\begin{equation}
\begin{split}
\min \ \ & \one^\top x\\
\text{s.t.} \ \ & \sum_{v \in V_i} x_v \geq 1 \text{ for all } i =1, \dots, l \\ 
& x_v \in \{0,1\} \text{ for all } v \in V
\end{split}
\end{equation}
where $x_v$ is a 0/1 variable that takes the value $1$ if we include item $j$ in set $S$ and $0$ if we do not include it in the set $S$.  
\end{general}



\todo[inline]{
Add flight crew scheduling example and images.
}

\begin{examplewithcode}{Capital Budgeting}{https://github.com/open-optimization/open-optimization-or-examples/blob/master/integer-programming/capital-budgeting.ipynb}
\noindent \textbf{Sets:}
\begin{itemize}
\item Let $I = \{1,2\}$ be the set of time periods.
\item Let $J = \{1, 2,3\}$ be the set of possible investments.
\end{itemize}

\noindent \textbf{Parameters:}
\begin{itemize}
\item $c_j$ is given in column "$\mathbb{E}$[Revenue]".
\item $b_i$ is given in row "Resources available".
\item $a_{ij}$ given in row $j$, and column for week $i$.
\end{itemize}

\noindent \textbf{Variables:}
\begin{itemize}
\item let $x_i = 0$ if investment $i$ is chosen
\item let $x_i = 1$ if investment $i$ is not chosen
\end{itemize}

The explict model is given by\\
\noindent  \textbf{Model:}
\begin{align*}
	\max\ \ \  & 10x_1+8x_2+6x_3 \tag{Total Expected Revenue}\\
	s.t. \ \ & 
	3x_1+1x_2+2x_3\leq5 \tag{ Resource constraint week 1}\\
	&4x_1+2x_2+1x_3\leq6 \tag{ Resource constraint week 2}\\
&x_j \in \{0,1\},\  j = 1,2,3
\end{align*}

\end{examplewithcode}




One specific type of set cover problem is the \emph{vertex cover} problem.
\begin{general}{Example: Vertex Cover}{\npcomplete}
\label{ex:vertex-cover}
Given a graph $G = (V,E)$ of vertices and edges, we want to find a smallest size subset $S \subseteq V$ such that every for every $e = (v,u) \in E$, either $u$ or $v$ is in $S$.   

We can write this as a mathematical program in the form:
\begin{equation}
\begin{split}
\min \ \ & \one^\top x\\
\text{s.t.} \ \ & x_u + x_v \geq 1 \text{ for all } (u,v) \in E \\ 
& x_v \in \{0,1\} \text{ for all } v \in V.
\end{split}
\end{equation}
\end{general}




\begin{examplewithcode}{Set cover: Fire station placement}{https://github.com/open-optimization/open-optimization-or-examples/blob/master/integer-programming/fire-station-covering.ipynb}
\label{example:fire-station}

In the fire station problem, we seek to choose locations for fire stations such that any district either contains a fire station, or neighbors a district that constains a fire station.  Figure~\ref{fig:tikz/Illustration1.pdf} depicts the set of districts and an example placement of locations of fire stations.  How can we minimize the the total number of fire stations that we need?


\noindent \textbf{Sets:}
\begin{itemize}
\item Let $V$ be the set of districts ($V = \{1, \dots, 16\}$)
\item Let $V_i$ be the set of districts that neighbor district $i$ (e.g. $V_1 = \{2,4,5\}$).
\end{itemize}

\noindent \textbf{Variables:}
\begin{itemize}
\item let $x_i = 1$ if district $i$ is chosen to have a fire station.
\item let $x_i = 0$ otherwise.
\end{itemize}
\textbf{Model:}
\begin{align}
\min  \  \ &\sum_{i \in V} x_i \tag{ \# \text{ open fire stations}}\\
\text{ s.t. }\ \ &  x_i + \sum_{j \in V_i} x_j \geq 1 & \forall{i \in V} \tag{Station proximity requirement}\\
& x_i \in \{0,1\} & \text{ for } i\in V  \tag{station either open or closed}
\end{align}
\end{examplewithcode}

\includefiguresource[Layout of districts and possible locations of fire stations.][][h]{tikz/Illustration1.pdf}
\includefiguresource[Set cover representation of fire station problem.  For example, choosing district 16 to have a fire station covers districts 13, 15, and 16.]{tikz/Illustration2.pdf}
\includefiguresource[Graph representation of fire station problem.  Every node is connected to a chosen node by an edge]{tikz/Illustration3.pdf}



%\documentclass[../open-optimization/open-optimization.tex]{subfiles}

%%%%%%%%%
%\begin{document}
%%%%%%%%%

\begin{general}{Set Covering - Matrix description}{\npcomplete}
\label{general:set-covering-alternate}
Given a non-negative matrix $A \in \{0,1\}^{m \times n}$, a non-negative vector, and an objective vector $c \in \R^n$, the set cover problem is
\begin{equation}
\begin{split}
\max \ \ & c^\top x\\
\text{s.t.}. \ \ & Ax \geq \one \\
& x \in \{0,1\}^n.
\end{split}
\end{equation}
\end{general}
\begin{examplewithoutcode}{Vertex Cover with matrix}{code:vertex-cover-matrix}
An alternate way to solve \nameref{ex:vertex-cover} is to define the \emph{adjacency matrix} $A$ of the graph.  The adjacency matrix is a $|E| \times |V|$ matrix with $\{0,1\}$ entries.  The each row corresponds to an edge $e$ and each column corresponds to a node $v$.  For an edge $e = (u,v)$, the corresponding row has a $1$ in columns corresponding to the nodes $u$ and $v$, and a 0 everywhere else.  Hence, there are exactly two 1's per row.  Applying the formulation above in \nameref{general:set-covering-alternate} models the problem.
\end{examplewithoutcode}




%%%%%%%%%
%\end{document}
%%%%%%%%%




\subsection{Covering (Generalizing Set Cover)}
We could also allow for a more general type of set covering where we have non-negative integer variables and a right hand side that has values other than $1$.
\begin{general}{Covering}{\npcomplete}
\label{general:covering}
Given a non-negative matrix $A \in \Z_+^{m \times n}$, a non-negative vector $b \in \Z^m$, and an objective vector $c \in \R^n$, the set cover problem is
\begin{equation}
\begin{split}
\max \ \ & c^\top x\\
\text{s.t.}. \ \ & Ax \geq b \\
& x \in\Z^n_+.
\end{split}
\end{equation}
\end{general}

%\todo[inline]{
% Add Nurse scheduling problem.
%}
%
%\begin{examplewithcode}{Nurse Scheduling}{code:nurse-scheduling}
%Suppose that we have $n$ full-time nurses employed and $p$ part-time nurses employed.  We need to schedule nurses such that, for each day of the week, we have enough nurses on hand to cover all the tasks.  
%
%\todo[inline]{Add complete example}
%\end{examplewithcode}




\section{Assignment Problem}
\todoSection{}
The \emph{assignment problem} (machine/person to job/task assignment) seeks to assign tasks to machines in a way that is most efficient.   This problem can be thought of as having a set of machines that can complete various tasks (textile machines that can make t-shirts, pants, socks, etc) that require different amounts of time to complete each task, and given a demand, you need to decide how to alloacte your machines to tasks.

Alternatively, you could be an employer with a set of jobs to complete and a list of employees to assign to these jobs.  Each employee has various abilities, and hence, can complete jobs in differing amounts of time.  And each employee's time might cost a different amout.  How should you assign your employees to jobs in order to minimize your total costs?


\begin{general}{Assignment Problem}{}{}
Given $m$ machines and $n$ jobs, find a least cost assignment of jobs to machines. The cost of assigning job $j$ to machine $i$ is $c_{ij}$.
\end{general}

\todo[inline]{
Include picture and example data
}


\begin{examplewithcode}{Machine Assignment}{https://github.com/open-optimization/open-optimization-or-examples/blob/master/integer-programming/assignment-problem.ipynb}
\noindent \textbf{Sets:}
\begin{itemize}
\item Let $I = \{0,1,2,3\}$ set of machines.
\item Let $J = \{0,1, 2,3\}$ be the set of tasks.
\end{itemize}

\noindent \textbf{Parameters:}
\begin{itemize}
\item $c_{ij}$ - the cost of assigning machine $i$ to job $j$
%\item $c_j$ is given in column "$\mathbb{E}$[Revenue]".
%\item $b_i$ is given in row "Resources available".
%\item $a_{ij}$ given in row $j$, and column for week $i$.
\end{itemize}

\noindent \textbf{Variables:}
\begin{itemize}
\item Let 
\begin{equation*}
x_{ij} = \begin{cases}
1 & \text{if machine $i$ assigned to job $j$}\\
0 & \text{otherwise.}
\end{cases}
\end{equation*}
%\item let $x_i = 1$ if investment $i$ is not chosen
\end{itemize}

\noindent  \textbf{Model:}
\begin{align*}
	\min\ \ \  & \sum_{i \in I, j \in J} c_{ij} x_{ij} \tag{Minimize cost}\\
	s.t. \ \ & \sum_{i \in I} x_{ij} = 1 & \text{for all $j \in J$}
	 \tag{All jobs are assigned one machine}\\
	& \sum_{j \in J} x_{ij} = 1 & \text{for all $i\in I$}
	 \tag{All machines are assigned to a job}\\
	 & x_{ij} \in \{0,1\} \forall i \in I, j \in J
	  \end{align*}
\end{examplewithcode}



\section{Facility Location}
\todoSection{}
The basic model of the facility location problem is to determine where to place your stores or facilities in order to be close to all of your customers and hence reduce the costs transportation to your customers.  Each customer is known to have a certain demand for a product, and each facility has a capacity on how much of that demand it can satisfy.  Furthermore, we need to consider the cost of building the facility in a given location.

This basic framework can be applied in many types of problems and there are a number of variants to this problem.   We will address two variants: the \emph{capacitated facility location problem} and the \emph{uncapacitated facility location problem}.  
\todo[inline]{
Add discussion on Facility Location Problems and pictures.
}



\subsection{Capacitated Facility Location}

\begin{general}{Capacitated Facility Location}{\npcomplete}
Given costs connections $c_{ij}$ and fixed building costs $f_i$, demands $d_j$ and capacities $u_i$, the capacitated facility location problem is 

\noindent \textbf{Sets:}
\begin{itemize}
\item Let $I = \{1,\dots, n\}$ be the set of facilities.
\item Let $J = \{1, \dots, m\}$ be the set of customers.
\end{itemize}

\noindent \textbf{Parameters:}
\begin{itemize}
\item $f_i$ - the cost of opening facility $i$.
\item $c_{ij}$ - the cost of fulfilling the complete demand of customer $j$ from facility $i$.
\item $u_{i}$ - the capacity of facility $i$.
\item $d_{j}$ - the demand by customer $j$.
%\item $c_j$ is given in column "$\mathbb{E}$[Revenue]".
%\item $b_i$ is given in row "Resources available".
%\item $a_{ij}$ given in row $j$, and column for week $i$.
\end{itemize}

\noindent \textbf{Variables:}
\begin{itemize}
\item Let 
\begin{equation*}
y_{i} = \begin{cases}
1 & \text{if we open facility $i$,}\\
0 & \text{otherwise.}
\end{cases}
\end{equation*}
\item Let $x_{ij} \geq 0$ be the fraction of demand of customer $j$ satisfied by facility $i$.
%\item let $x_i = 1$ if investment $i$ is not chosen
\end{itemize}

\noindent  \textbf{Model:}
\begin{align}
\min & \displaystyle\sum_{i=1}^n\sum_{j=1}^mc_{ij}x_{ij}+\sum_{i=1}^nf_iy_i \tag{total cost}\\
\text{s.t.} & \displaystyle\sum_{i=1}^nx_{ij}=1 \text{ for all }j=1,\dots,m \tag{assign demand to facility}\\
& \displaystyle \sum_{j=1}^md_jx_{ij}\leqslant u_iy_i\text{ for all }i=1\dots,n \tag{capacity of facility $i$}\\
&x_{ij}\geqslant0\text{ for all }i=1,\dots,n \text{ and }j=1,\dots,m \tag{nonnegative fraction of demand satisfied}\\
&y_i\in\{0,1\}\text{ for all } i=1,\dots,n \tag{open/not open facility}
\end{align}

\end{general}

Alternative model!

\begin{align*}
\min \ \ \ & \sum_{i = 1}^m\sum_{j=1}^n c_{ij}x_{ij}+\sum_{i = 1}^mf_iy_i\\
s.t.&  \sum_{i = 1}^mx_{ij} = 1,~~~~~j =1,\ldots n\\
& \displaystyle \sum_{j=1}^md_jx_{ij}\leqslant u_i\text{ for all }i=1\dots,n \tag{capacity of facility $i$}\\
&x_{ij}\leq y_i,~~~~~ i = 1,\ldots m,~~~ j = 1,\ldots,n\\
&  x_{ij} \geq 0~~~~~ i = 1,\ldots,m, ~~~j = 1,\ldots,n\\
& y_i\in \{0,1\}, \  \  i = 1,\ldots,m
\end{align*}

\subsection{Uncapacitated Facility Location}
\begin{general}{Uncapacitated Facility Location}{\npcomplete}
Given costs connections $c_{ij}$ and fixed building costs $f_i$, the uncapacitated facility location problem is 
\begin{equation}
\begin{array}{rl}
\min & \displaystyle\sum_{i=1}^n\sum_{j=1}^mc_{ij}z_{ij}+\sum_{i=1}^nf_ix_i \\
\text{s.t.} & \displaystyle\sum_{i=1}^nz_{ij}=1 \text{ for all }j=1,\dots,m \\
& \displaystyle \sum_{j=1}^mz_{ij}\leqslant Mx_i\text{ for all }i=1\dots,n \\
&z_{ij}\in\{0,1\}\text{ for all }i=1,\dots,n \text{ and }j=1,\dots,m\\
&x_i\in\{0,1\}\text{ for all } i=1,\dots,n
\end{array}
\end{equation}
Here $M$ is a large number and can be chosen as $M = m$, but could be refined smaller if more context is known.
\end{general}

Alternative model!

\begin{equation}
\begin{array}{rl}
\min & \displaystyle\sum_{i=1}^n\sum_{j=1}^mc_{ij}z_{ij}+\sum_{i=1}^nf_ix_i \\
\text{s.t.} & \displaystyle\sum_{i=1}^nz_{ij}=1 \text{ for all }j=1,\dots,m \\
& z_{ij}\leqslant x_i\text{ for all }i=1\dots,n \text{ for all } j = 1, \dots, m \\
&z_{ij}\in\{0,1\}\text{ for all }i=1,\dots,n \text{ and }j=1,\dots,m\\
&x_i\in\{0,1\}\text{ for all } i=1,\dots,n
\end{array}
\end{equation}
\newpage
\section{Basic Modeling Tricks - Using Binary Variables}
%\todoSection{}
In this section, we describe ways to model a variety of constraints that commonly appear in practice.  The goal is changing constraints described in words to constraints defined by math.

Binary variables can allow you to model many types of constraints.  We discuss here varios logical constraints where we assume that $x_i \in \{0,1\}$ for $i=1, \dots, n$.  We will take the meaning of the variable to be selecting an item.

\begin{enumerate}
 \item If item $i$ is selected, then item $j$ is also selected.
 \begin{equation}
 x_i \leq x_j
 \end{equation}
 \begin{enumerate}
 \item If any of items $1, \dots, 5$ are selected, then item $6$ is selected.
 \begin{equation}
 x_1 + x_2 + \dots + x_5 \leq 5 \cdot x_6
 \end{equation}
 Alternatively!
 \begin{equation}
 x_i \leq x_6 \ \ \ \ \text{ for all } i=1, \dots, 5
 \end{equation}
 \end{enumerate}
 \item If item $j$ is not selected, then item $i$ is not selected.
 \begin{equation}
x_i \leq x_j
 \end{equation}
  \begin{enumerate}
 \item If item $j$ is not selected, then all items $1, \dots, i$ are not selected.
 \begin{equation}
 x_1 + x_2 + \dots + x_i \leq i \cdot x_j
 \end{equation}
 \end{enumerate}
  \item If item $j$ is not selected, then item $i$ is not selected.
 \begin{equation}
x_i \leq x_j
 \end{equation}
\item Either item $i$ is selected or item $j$ is selected, but not both.
 \begin{equation}
 x_i + x_j = 1
 \end{equation}
\item Item $i$ is selected or item $j$ is selected or both.
 \begin{equation}
 x_i + x_j \geq 1
 \end{equation}
\item If item $i$ is selected, then item $j$ is not selected.
 \begin{equation}
x_j \leq (1-x_i)
 \end{equation}

\item At most one of items $i ,j$, and $k$ are selected.
 \begin{equation}
 x_i + x_j + x_k \leq 1
 \end{equation}
\item At most two of items $i,j,$ and $k$ are selected.
 \begin{equation}
 x_i + x_j + x_k \leq 2
 \end{equation}
\item Exactly one of items $i,j,$ and $k$ are selected.
 \begin{equation}
  x_i + x_j + x_k = 1
 \end{equation}
\end{enumerate}

These tricks can be connected to create different function values.  

\begin{example}{Variable takes one of three values}{}
Suppose that the variable $x$ should take one of the three values $\{4, 8, 13\}$. This can be modeled using three binary variables as
\begin{align*}
x=4z_1 +8z_2 +13z_3\\
 z_1 +z_2 +z_3 =1\\
z_i \in \{0, 1\} \text{ for } i = 1, 2,3.
\end{align*}
As a convenient addition, if we want to add the possibility that it takes the value $0$, then we can model this as 
\begin{align*}
x=4z_1 +8z_2 +13z_3\\
 z_1 +z_2 +z_3 \leq 1\\
z_i \in \{0, 1\} \text{ for } i = 1, 2,3.
\end{align*}
\end{example}


We can also model variable increases at different amounts.  \begin{example}{Discount for buying more}{}
Suppose you can choose to buy 1, 2, or 3 units of a product, each with a decreasing cost.  The first unit is \$10, the second is \$5, and the third unit is \$3.
\begin{align*}
x=10z_1 +5z_2 +3z_3\\
 z_1\geq z_2 \geq z_3\\
z_i \in \{0, 1\} \text{ for } i = 1, 2,3.
\end{align*}
Here, $z_i$ represents if we buy the $i$th unit.  The inequality constraints impose that if we buy unit $j$, then we must by all units $i$ with $i < j$.
\end{example}



In this section, we describe ways to model a variety of constraints that commonly appear in practice.  The goal is changing constraints described in words to constraints defined by math.
\subsection{Big M constraints - Activating/Deactivating Inequalities}
Big M comes again!   It's extremely useful when trying to activate constrants based on a binary variable.

For intance, if we don't rent a bus, then we can have at most 3 passengers join us on our trip.   Consider passengers $A,B,C,D,E$ and let $x_i \in \{0,1\}$ be 1 if we take passenger $i$ and 0 otherwise.
We can model the constraint that we can have at most 5 passengers as 
$$
x_A + x_B + x_C + x_D + x_E \leq 3.
$$

We want to be able to activate this constraint in the event that we don't rent a bus.

Let $\delta \in \{0,1\}$ be 1 if rent a bus, and 0 otherwise.  

Then we want to say
\begin{quote}
If $\delta = 0$ , then 
$$
x_A + x_B + x_C + x_D + x_E \leq 3.
$$
\end{quote}

We can formulate this using a big-M constraint as

\begin{equation}
x_A + x_B + x_C + x_D + x_E \leq 3 + M \delta.
\end{equation}

Notice the two case
$$
\begin{cases}
x_A + x_B + x_C + x_D + x_E \leq 3  & \text{ if } \delta = 0\\
x_A + x_B + x_C + x_D + x_E \leq 3 +M & \text{ if } \delta = 1
\end{cases}
$$
In the second case, we choose $M$ to be so large, that the second case inequality is vacuous.    That said, choosing smaller $M$ values (that are still valid) will help the computer program solve the problem faster.  In this case, it suffices to let $M = 2$.

We can speak about this technique more generally as
\begin{general}{Big-M: If then}{}{}
We aim to model the relationship
\begin{equation}
\text{ If } \ \ \   \delta = 0, \ \ \  \text{then  } \ \ \ a^\top x \leq b.
\end{equation}
By letting $M$ be an upper bound on the quantity $a^\top x - b$, we can model this condition as 
\begin{equation}
\begin{split}
a^\top x - b& \leq M\delta\\
\delta & \in \{0,1\}
\end{split}
\end{equation}
\end{general}


\subsection{Either Or Constraints}
\emph{``At least one of these constraints holds"} is what we would like to model.  Equivalently, we can phrase this as an \emph{inclusive or} constraint.  This can be molded with a pair of Big-M constraints.
\begin{general}{Either Or}{}{}

\begin{equation}
\text{Either} \ \ \ a^\top x \leq b\ \  \text{  or  } \ \  c^\top x \leq d \ \ \text{holds} 
\end{equation}
can be modeled as 
\begin{equation}
\begin{split}
a^\top x - b &\leq M_1 \delta\\
c^\top x - d &\leq M_2 (1-\delta)\\
\delta &\in \{0,1\},
\end{split}
\end{equation}
where $M_1$ is an upper bound on $a^\top x - b$ and $M_2$ is an upper bound on $c^\top x - d$.
\end{general}

\begin{example}{}{}
Either $2$ buses or $10$ cars are needed shuttle students to the football game.  
\begin{itemize}
\item Let $x$ be the number of buses we have and 
\item let $y$ be the number of cars that we have.  
\end{itemize}
Suppose that there are at most $M_1 = 5$ buses that could be rented an at most $M_2 = 20$ cars that could be available.

This constraint can be modeled as 
\begin{equation}
\begin{split}
x - 2 &\leq 5\delta\\
y - 10 &\leq 20 (1-\delta)\\
\delta &\in \{0,1\},
\end{split}
\end{equation}
\end{example}
\subsection{If then implications - opposite direction}
Suppose that we want to model the fact that if we have at most 10 students attending this course, then we must switch to a smaller classroom.   

Let $x_i \in \{0,1\}$ be 1 if student $i$ is in the course or not.   Let $\delta \in \{0,1\}$ be 1 if we need to switch to a smaller classroom.  

Thus, we want to model 
\begin{quote}
If $$
\sum_{i \in I} x_i \leq 10
$$
then 
$$
\delta = 1.
$$
\end{quote}

We can model this as 
\begin{equation}
\sum_{i \in I} x_i \geq 10 + 1 + M\delta.
\end{equation}


\begin{general}{If inequality, then indicator}
We let $m$ be a lower bound on the quantity $a^\top x - b$ and we let $\epsilon$ be a tiny number that is an error bound in verifying if an inequality is violated.  \textbf{If the data $a,b$ are integer and $x$ is an integer, then we can take $\epsilon = 1$.}

Now
\begin{equation}
\text{If } \ \ a^\top x \leq b  \ \ \text{then}\ \ \delta = 1
\end{equation}
can be modeled as 
\begin{equation}
a^\top x -b  \geq  \epsilon(1-\delta) + m \delta.
\end{equation}
\end{general}



\begin{proof}We not justify the statement above.

A simple way to understand this constraint is to consider the \emph{contrapositive} of the if then statement that we want to model.  The contrapositive says that 
\begin{equation}
\text{If $\delta = 0$, then $a^\top x - b > 0$.}
\end{equation}
To show the contrapositive, we set $\delta = 0$.  Then the inequality becomes 
$$
a^\top x - b \geq \epsilon(1-0) + m0 = \epsilon > 0.
$$
Thus, the contrapositive holds.

\textbf{If instead we wanted a direct proof:}

Case 1: Suppose $a^\top x \leq b$.  Then $0 \geq a^\top x - b$, which implies that 
$$
\delta(a^\top x - b) \geq a^\top x - b
$$
Therefore
$$
\delta(a^\top x - b) \geq \epsilon(1-\delta) + m \delta
$$
After rearranging
$$
\delta(a^\top x - b - m) \geq \epsilon(1-\delta)
$$
Since $a^\top x - b - m \geq 0$ and $\epsilon > 0$, the only feasible choice is $\delta = 1$.


Case 2:  Suppose $a^\top x > b$.  Then $a^\top x - b \geq \epsilon$.    Since $a^\top x -b \geq m$, both choices $\delta = 0$ and $\delta = 1$ are feasible.

By the choice of $\epsilon$,  we know that $a^\top x -b > 0$ implies that $a^\top x - b \geq \epsilon$.  



Since we don't like strict inequalities, we write the strict inequality as $a^\top x - b \geq \epsilon$ where $\epsilon$ is a small positive number that is a smallest difference between $a^\top x - b$ and $0$ that we would typically observe.  As mentioned above, if $a,b,x$ are all integer, then we can use $\epsilon = 1$.

Now we want an inequality with left hand side $a^\top x - b \geq$ and right hand side to take the value 
\begin{itemize}
\item $\epsilon$ if $\delta = 0$,
\item $m$ if $\delta = 1$.
\end{itemize}
This is accomplished with right hand side $\epsilon (1-\delta) + m\delta$.
\end{proof}

Many other combinations of if then statements are summarized in the following table:
\begin{table}
\begin{center}
\begin{tabular}{|c|c|}
\hline
\textbf{Implication} & \textbf{Constraint}\\
\hline
If $\delta = 0$, then $a^\top x \leq b$ & $a^\top x \leq b + M \delta$\\
If $a^\top x \leq b$, then $\delta = 1$ & $a^\top x \geq m \delta + \epsilon(1-\delta)$\\
\hline
\end{tabular}
\end{center}
\caption{Short list: If/then models with a constraint and a binary variable.  Here $M$ and $m$ are upper and lower bounds on $a^\top x - b$ and $\epsilon$ is a small number such that if $a^\top x > b$, then $a^\top x \geq b + \epsilon$.}
\end{table}
These two implications can be used to derive the following longer list of implications.

\begin{table}
\begin{center}
\begin{tabular}{|c|c|}
\hline
\textbf{Implication} & \textbf{Constraint}\\
\hline
If $\delta = 0$, then $a^\top x \leq b$ & $a^\top x \leq b + M \delta$\\
If $\delta = 0$, then $a^\top x \geq b$ & $a^\top x \geq b + m \delta$\\
If $\delta = 1$, then $a^\top x \leq b$ & $a^\top x \leq b + M (1-\delta)$\\
If $\delta = 1$, then $a^\top x \geq b$ & a$^\top x \geq b + m (1-\delta)$\\
If $a^\top x \leq b$, then $\delta = 1$ & $a^\top x \geq b + m \delta + \epsilon(1-\delta)$\\
If $a^\top x \geq b$, then $\delta = 1$ & $a^\top x \leq b + M \delta - \epsilon(1-\delta)$\\
If $a^\top x \leq b$, then $\delta = 0$ & $a^\top x \geq b + m (1-\delta) + \epsilon \delta$\\
If $a^\top x \geq b$, then $\delta = 0$ & $a^\top x \geq b + m (1-\delta) - \epsilon \delta$\\
\hline
\end{tabular}
\end{center}
\caption{Long list: If/then models with a constraint and a binary variable.  Here $M$ and $m$ are upper and lower bounds on $a^\top x - b$ and $\epsilon$ is a small number such that if $a^\top x > b$, then $a^\top x \geq b + \epsilon$.}
\end{table}
Lastly, if you insist on having exact correspondance, that is, "$\delta = 0 $ if and only if $a^\top x \leq b$" you can simply include both constraints for "if $\delta = 0 $, then $a^\top x \leq b$" and if "$a^\top x \leq b$, then $\delta = 0 $".  Although many problems may be phrased in a way that suggests you need "if and only if", it is often not necessary to use both constraints due to the objectives in the problem that naturally prevent one of these from happening.  

For example, if we want to add a binary variable $\delta$ that means
$$
\begin{cases}
\delta = 0 \text{ implies }  a^\top x \leq b\\
\delta = 1  \text{ Otherwise}
\end{cases}
$$
If $\delta = 1$ does not effect the rest of the optimization problem, then adding the constraint regarding $\delta = 1$ is not necessary.  Hence, typically, in this scenario, we only need to add the constraint $a^\top x \leq b + M \delta$.

\subsection{Binary reformulation of integer variables}
If an integer variable has small upper and lower bounds, it can sometimes be advantageous to recast it as a sequence of binary variables - for either modeling, the solver, or both.   Although there are technically many ways to do this, here are the two most common ways.

\begin{general}{Full reformulation}{\textcolor{blue}{$u$ many binary variables}}
\label{general:full-reformulation}
For a non-negative integer variable $x$ with upper bound $u$, modeled as 
\begin{equation}
0 \leq x \leq u, \ \ \ \ x \in \Z,
\end{equation}
this can be reformulated with $u$ binary variables $z_1, \dots, z_u$ as 
\begin{equation}
\begin{split}
x & = \sum_{i=1}^u i z_i = z_1 + 2 z_2 + \dots + u z_u\\
1 & \geq \sum_{i=1}^u z_i = z_1 + z_2 + \dots + z_u\\
z_i & \in \{0,1\} \ \ \text{ for } i=1, \dots, u
\end{split}
\end{equation}
\end{general}
We call this the \emph{full reformulation} because there is a binary variable $z_i$ associated with every value $i$ that $x$ could take.  That is, if $z_3 = 1$, then the second constraint forces $z_i = 0$ for all $i \neq 3$ (that is, $z_3$ is the only non-zero binary variable), and hence by the first constraint, $x = 3$.

\begin{general}{Binary reformulation}{\textcolor{blue}{$O(\log u)$ many binary variables}}
\label{general:log-reformulation}
For a non-negative integer variable $x$ with upper bound $u$, modeled as 
\begin{equation}
0 \leq x \leq u, \ \ \ \ x \in \Z,
\end{equation}
this can be reformulated with $u$ binary variables $z_1, \dots, z_{\lfloor\log( u )\rfloor}$ as 
\begin{equation}
\begin{split}
x & = \sum_{i=0}^{\lfloor\log( u )\rfloor}2^i z_i = z_0 + 2 z_1 +  4 z_2 + 8 z_3 + \dots + 2^{\lfloor\log( u )\rfloor} z_{\lfloor\log( u )\rfloor}\\
z_i & \in \{0,1\} \ \ \text{ for } i=1, \dots, \lfloor\log( u )\rfloor
\end{split}
\end{equation}
\end{general}
We call this the \emph{log reformulation} because this requires only logarithmically many binary variables in terms of the upper bound $u$.   This reformulation is particularly better than the full reformulation when the upper bound $u$ is a ``larger" number, although we will leave it ambiguous as to how larger a number need to be in order to be described as a ``larger" number. 
\subsection{SOS1 Constraints}
\begin{definition}{Special Ordered Sets of Type 1 (SOS1)}
A Special Ordered Sets of type 1 (SOS1) constraint on a vector indicates that \emph{at most one element of the vector can non-zero}.
\end{definition}
We next give an example of how to use binary variables to model this and then show how much simpler it can be coded using the SOS1 constraint.

\begin{examplewithcode}{SOS1 Constraints}{https://github.com/open-optimization/open-optimization-or-examples/blob/master/nonlinear-programming/Gurobi-SOS-Examples.ipynb}
\label{example:sos1}
Solve the following optimization problem: \[\begin{aligned}
\text{maximize}\quad & 3x_1 + 4x_2 + x_3 + 5x_4 \\
\text{subject to}\quad & 0 \le x_i \le 5 \\
& \text{at most one of the $x_i$ can be nonzero}
\end{aligned}\]
\end{examplewithcode}
\subsection{SOS2 Constraints}
\begin{definition}{Special Ordered Sets of Type 2 (SOS2)}{}
A \emph{Special Ordered Set of Type 2 (SOS2)} constraint on a vector indicates that \emph{at most two elements of the vector can non-zero AND the non-zero elements must appear consecutively}.
\end{definition}
We next give an example of how to use binary variables to model this and then show how much simpler it can be coded using the SOS2 constraint.
\begin{examplewithcode}{SOS2}{https://github.com/open-optimization/open-optimization-or-examples/blob/master/nonlinear-programming/Gurobi-SOS-Examples.ipynb}
\label{example:SOS2}
Solve the following optimization problem: \[\begin{aligned}
\text{maximize}\quad & 3x_1 + 4x_2 + x_3 + 5x_4 \\
\text{subject to}\quad & 0 \le x_i \le 5 \\
& \text{at most two of the $x_i$ can be nonzero} \\
& \text{and the nonzero $x_i$ must be consecutive}
\end{aligned}\]
\end{examplewithcode}

\subsection{Piecewise linear functions with SOS2 constraint}
\begin{examplewithcode}{Piecewise Linear Function}{https://github.com/open-optimization/open-optimization-or-examples/blob/master/nonlinear-programming/Gurobi-SOS-Examples.ipynb}
\label{example:pwl}
Consider the piecewise linear function 
 $c(x)$ given by
$$
c(x) = 
\begin{cases}
25x  & \text{ if } 0 \leq x \leq 5\\
20x + 25 & \text{ if } 5 \leq x \leq 10\\
15x + 75 & \text{ if } 10 \leq x \leq 15
\end{cases}
$$

\includegraphicstatic[scale = 0.35]{pwl-plot.png}

We will use integer programming to describe this function.  We will fix $x = a$ and then the integer program will set the value $y$ to $c(a)$.
\begin{align*}
\min\quad & 0\\
\text{Subject to} \quad & x - 5 z_{2} - 10 z_{3} - 15 z_{4} = 0\\
 & y - 125 z_{2} - 225 z_{3} - 300 z_{4} = 0\\
 & z_{1} + z_{2} + z_{3} + z_{4} = 1\\
 & SOS2: \{z_1, z_2, z_3, z_4\}\\
 & 0 \leq z_{i} \leq 1 \quad\forall i \in \{1,2,3,4\}\\
 & x = a\\
\end{align*}

\end{examplewithcode}

\begin{examplewithcode}{Piecewise Linear Function Application}{code:pwl-application}
\label{example:pwl-application}
Consider the following optimization problem where the objective function includes the term $c(x)$, where $c(x)$ is the piecewise linear function described in \nameref{example:pwl}:
\begin{align}
\max & z = 12x_{11} + 12x_{21} + 14x_{12} + 14x_{22} - c(x)\\
\text{s.t.} & x_{11} + x_{12} \leq x + 5\\
& x_{21} + x_{22} \leq 10\\
& 0.5 x_{11} - 0.5x_{21} \geq 0\\
& 0.4 x_{12} - 0.6 x_{22} \geq 0\\
& x_{ij} \geq 0\\
& 0 \leq x \leq 15
\end{align}

Given the piecewise linear, we can model the whole problem explicitly as a mixed-integer linear program.

 \begin{equation}
 \begin{array}{rrlr}
 \max\quad & 12 X_{1,1} + 12 X_{2,1} + 14 X_{1,2} + 14 X_{2,2} - y\\
\text{Subject to} \quad & x - 5 z_{2} - 10 z_{3} - 15 z_{4} &= 0\\
 & y - 125 z_{2} - 225 z_{3} - 300 z_{4} &= 0\\
 & z_{1} + z_{2} + z_{3} + z_{4} &= 1\\
 & X_{1,1} + X_{1,2} - x &\leq 5\\
 & X_{2,1} + X_{2,2} \leq 10\\
 & 0.5 X_{1,1} - 0.5 X_{2,1} &\geq 0\\
 & 0.4 X_{1,2} - 0.6 X_{2,2} &\geq 0\\
 & SOS2: \{z_1, z_2, z_3, z_4\}\\
 & X_{i,j} &\geq 0 &\quad\forall i \in \{1,2\}, j \in \{1,2\}\\
 & 0 \leq z_{i} &\leq 1 &\quad\forall i \in \{1,2,3,4\}\\
 & 0 \leq x &\leq 15\\
 & y & \text{ free} \\
\end{array}
\end{equation}
\end{examplewithcode}


\subsubsection{SOS2 with binary variables}

{Modeling Piecewise linear function}
{\begin{itemize}
\item Write down pairs of breakpoints and functions values $(a_i, f(a_i))$.
\item Define a binary variable $z_i$ indicating if $x$ is in the interval $[a_i, a_{i+1}]$.
\item Define multipliers $\lambda_i$ such that $x$ is a combination of the $a_i$'s and therefore the output $y = f(x)$ is a combination of the $f(a_i)$'s.
\item Restrict that at most 2 $\lambda_i's$ are non-zero and that those 2 are consecutive.

\end{itemize}

\begin{align*}
\min \ \ & \sum_{i=1}^k \lambda_i f(a_i)\\
\text{s.t.} \ \ &  \sum_{i=1}^k \lambda_i = 1\\
& x = \sum_{i=1}^k \lambda_i a_i\\
& \lambda_1 \leq z_1\\
& \lambda_i \leq z_{i-1} + z_{i} & \text{ for }  i=2, \dots, k-1,\\
& \lambda_k \leq z_{k-1}\\
& \lambda_i \geq 0, y_i \in \{0,1\}.
\end{align*}

\subsection{Maximizing a minimum}
When the constraints could be general, we will write $x \in X$ to define general constraints.  For instance, we could have $X = \{ x \in \R^n : Ax \leq b\}$ of $X  = \{ x \in \R^n : Ax \leq b, x \in \Z^n\}$ or many other possibilities.  


Consider the problem 

\begin{align*}
\max   \quad & \min \{x_1, \dots, x_n\}\\
\text{ such that } \quad &  x \in X
\end{align*}
Having the minimum on the inside is inconvenient.  To remove this, we just define a new variable $y$ and enforce that $y \leq x_i$ and then we maximize $y$.  Since we are maximizing $y$, it will take the value of the smallest $x_i$.  Thus, we can recast the problem as

\begin{align*}
\max\quad    & y\\
\text{ such that } \quad  & y \leq x_i \ \text{ for }\  i=1, \dots, n \\
&  x \in X
\end{align*}


\subsection{Relaxing (nonlinear) equality constraints}

There are a number of scenarios where the constraints can be relaxed without sacrificing optimal solutions to your problem.   In a similar vein of the maximizing a minimum, if because of the objective we know that certain constraints will be tight at optimal solutions, we can relax the equality to an inequality.   For example, 
\begin{align*}
\max   \quad &x_1 + x_2 +  \dots + x_n\\
\text{ such that } \quad &  x_i = y_i^2 + z_i^2 \text{ for } i=1, \dots, n
\end{align*}







\subsection{Connecting to continuous variables}

Let $x_i \geq 0$ and $y_i \in \{0,1\}$ for all $i=1, \dots, n$.

\begin{enumerate}
\item If $x_i > 0$, then $y_i = 1$.
\begin{equation}
x_i \leq M y_i
\end{equation}
where $M$ is a suffiently large upper bound on the variable $x_i$.
\item If $x_i = 0$, then $y_i = 0$.\\
This is harder to model!  Alternatively, we try modeling "if $x_i$ is sufficiently small, then $y_i = 0$.   For instance, if $x_i \leq 0.0000001$, then $y_i = 0$.
This can be modeled as
\begin{equation}
x_i - 0.0000001 \geq y_i -1.
\end{equation}
\item If $y_i = 1$, then $x_i \geq 5$
\begin{equation}
5y_i \leq x_i.
\end{equation}
\end{enumerate}

\subsection{Exact absolute value}
% Borrowed from Mosek Cookbook
Suppose we need to model an exact equality
$$
|x|=t
$$
It defines a non-convex set, hence it is not conic representable. If we split $x$ into positive and negative part $x=x^{+}-x^{-}$, where $x^{+}, x^{-} \geq 0$, then $|x|=x^{+}+x^{-}$as long as either $x^{+}=0$ or $x^{-}=0$. That last alternative can be modeled with a binary variable, and we get a model of  :
$$
\begin{aligned}
x &=x^{+}-x^{-} \\
t &=x^{+}+x^{-} \\
0 & \leq x^{+}, x^{-} \\
x^{+} & \leq M z \\
x^{-} & \leq M(1-z) \\
z & \in\{0,1\}
\end{aligned}
$$
where the constant $M$ is an a priori known upper bound on $|x|$ in the problem.

\subsubsection{Exact 1 -norm}
We can use the technique above to model the exact $\ell_{1}$-norm equality constraint
$$
\sum_{i=1}^{n}\left|x_{i}\right|=c
$$
where $x \in \mathbb{R}^{n}$ is a decision variable and $c$ is a constant. Such constraints arise for instance in fully invested portfolio optimizations scenarios (with short-selling). As before, we split $x$ into a positive and negative part, using a sequence of binary variables to guarantee that at most one of them is nonzero:
$$
\begin{aligned}
x &=x^{+}-x^{-} \\
0 & \leq x^{+}, x^{-}, \\
x^{+} & \leq c z \\
x^{-} & \leq c(e-z), \\
\sum_{i} x_{i}^{+}+\sum_{i} x_{i}^{-} &=c, \\
z & \in\{0,1\}^{n}, x^{+}, x^{-} \in \mathbb{R}^{n}
\end{aligned}
$$

\subsubsection{Maximum}
The exact equality $t=\max \left\{x_{1}, \ldots, x_{n}\right\}$ can be expressed by introducing a sequence of mutually exclusive indicator variables $z_{1}, \ldots, z_{n}$, with the intention that $z_{i}=1$ picks the variable $x_{i}$ which actually achieves maximum. Choosing a safe bound $M$ we get a model:
$$
\begin{aligned}
x_{i} & \leq t \leq x_{i}+M\left(1-z_{i}\right), i=1, \ldots, n \\
z_{1}+\cdots+z_{n} &=1, \\
z & \in\{0,1\}^{n}
\end{aligned}
$$

\newpage







\section{Network Flow}
\todoSection{}
\todo[inline]{Fix up this section}
\begin{center}
\href{https://pixabay.com/illustrations/under-construction-construction-sign-2408060/}{\includegraphics[scale = 0.05]{optimization/figures/under-construction-2408060_1280}}
\end{center}
\subsection{Example - Multicommodity Flow}

\url{https://en.wikipedia.org/wiki/Multi-commodity_flow_problem}
The \textbf{multi-commodity flow problem} is a
\href{flow_network}{network flow} problem with multiple commodities
(flow demands) between different source and sink nodes.

\paragraph{Problem Definition}

Given a flow network \(\,G(V,E)\), where edge \((u,v) \in E\) has
capacity \(\,c(u,v)\). There are \(\,k\) commodities
\(K_1,K_2,\dots,K_k\), defined by \(\,K_i=(s_i,t_i,d_i)\), where
\(\,s_i\) and \(\,t_i\) is the \textbf{source} and \textbf{sink} of
commodity \(\,i\), and \(\,d_i\) is its demand. The variable
\(\,f_i(u,v)\) defines the fraction of flow \(\,i\) along edge
\(\,(u,v)\), where \(\,f_i(u,v) \in [0,1]\) in case the flow can be
split among multiple paths, and \(\,f_i(u,v) \in \{0,1\}\) otherwise
(i.e. "single path routing"). Find an assignment of all flow variables
which satisfies the following four constraints:

\textbf{(1) Link capacity:} The sum of all flows routed over a link does
not exceed its capacity.

\[\forall (u,v)\in E:\,\sum_{i=1}^{k} f_i(u,v)\cdot d_i \leq c(u,v)\]

\textbf{(2) Flow conservation on transit nodes:} The amount of a flow
entering an intermediate node \(u\) is the same that exits the node.

\[\,\sum_{w \in V} f_i(u,w) - \sum_{w \in V} f_i(w,u) = 0 \quad \mathrm{when} \quad u \neq s_i, t_i\]

\textbf{(3) Flow conservation at the source:} A flow must exit its
source node completely.

\[\,\sum_{w \in V} f_i(s_i,w) - \sum_{w \in V} f_i(w,s_i) = 1\]

\textbf{(4) Flow conservation at the destination:} A flow must enter its
sink node completely.

\[\,\sum_{w \in V} f_i(w,t_i) - \sum_{w \in V} f_i(t_i,w) = 1\]

\hypertarget{corresponding-optimization-problems}{%
\subsection{Corresponding optimization
problems}\label{corresponding-optimization-problems}}

\textbf{Load balancing} is the attempt to route flows such that the
utilization \(U(u,v)\) of all links \((u,v)\in E\) is even, where

\[U(u,v)=\frac{\sum_{i=1}^{k} f_i(u,v)\cdot d_i}{c(u,v)}\]

The problem can be solved e.g. by minimizing
\(\sum_{u,v\in V} (U(u,v))^2\). A common linearization of this problem
is the minimization of the maximum utilization \(U_{max}\), where

\[\forall (u,v)\in E:\, U_{max} \geq U(u,v)\]

In the \textbf{minimum cost multi-commodity flow problem}, there is a
cost \(a(u,v) \cdot f(u,v)\) for sending a flow on \(\,(u,v)\). You then
need to minimize

\[\sum_{(u,v) \in E} \left( a(u,v) \sum_{i=1}^{k} f_i(u,v) \right)\]

In the \textbf{maximum multi-commodity flow problem}, the demand of each
commodity is not fixed, and the total throughput is maximized by
maximizing the sum of all demands \(\sum_{i=1}^{k} d_i\)

\hypertarget{relation-to-other-problems}{%
\subsection{Relation to other
problems}\label{relation-to-other-problems}}

The minimum cost variant of the multi-commodity flow problem is a
generalization of the \href{minimum_cost_flow_problem}{minimum cost flow
problem} (in which there is merely one source \(s\) and one sink \(t\).
Variants of the \href{circulation_problem}{circulation problem} are
generalizations of all flow problems. That is, any flow problem can be
viewed as a particular circulation problem.\footnote{}

\hypertarget{usage}{%
\subsection{Usage}\label{usage}}

\href{Routing_and_wavelength_assignment}{Routing and wavelength
assignment} (RWA) in \href{optical_burst_switching}{optical burst
switching} of \href{SONET}{Optical Network} would be approached via
multi-commodity flow formulas.


\section{Transportation Problem}
\todoSection{}
\todo[inline]{
Add discussion of transportation problem and picture.
}

\href{https://www.youtube.com/watch?v=Jr7LI-sUEmo}{Youtube! - TRANSPORTATION PROBLEM with PuLP in PYTHON}

\href{https://nbviewer.jupyter.org/github/Pyomo/PyomoGallery/blob/master/transport/transport.ipynb}{Notebook: Solution with Pyomo}

\section{Job Shop Scheduling}
\todoSection{}
\href{https://python-mip.readthedocs.io/en/latest/examples.html#job-shop-scheduling-problem}{Pyhton MIP example}
\todo[inline]{Fill in model and discussion and add code example.  Need to create gnat chart code for nice visulizations. }


\section{Jobshop Scheduling: Makespan Minimization}
\todo[inline]{
 Add discussion of some makespan minimization problems.
}

\href{https://en.wikipedia.org/wiki/Job_shop_scheduling}{Wikipedia: Jobshop Scheduling}
$$
x_{ij} = \text{ start time of job $j$ on machine $i$}.
$$

$$
y_{ijk} = \begin{cases} 1, & \text{if job } j \text{ precedes job } k \text{ on machine } i \text{,}\\ & i \in I \text{, } j, k  \in J \text{, } j \neq k \\ 0, & \text{otherwise} \end{cases}
$$



\begin{align}\textrm{min: }  &  \\
               & C \\
\textrm{s.t.: } &  \\
               x_{o^{j}_{r}j} &  \geq x_{o^{j}_{r-1}j} +p_{o^{j}_{r-1}j} \,\,\, \forall r \in \{2,..,m\}, j \in {J} \\
                   x_{ij}     & \geq x_{ik} + p_{ik} - M \cdot y_{ijk} \,\,\, \forall j,k \in {J}, j \neq k, i \in {M} \\
                   x_{ik}     & \geq x_{ij} + p_{ij} - M \cdot (1-y_{ijk}) \,\,\, \forall j,k \in {J}, j \neq k,i \in {I} \\
                   C          & \geq x_{o^{j}_{m}j} + p_{o^{j}_{m}j} \,\,\, \forall j \in {J} \\
                  x_{ij}      & \geq 0 \,\,\, \forall i \in {J}, i \in {I} \\
                  y_{ijk}     & \in \{0,1\} \,\,\, \forall j,k \in {J}, i \in {I} \\
                  C & \geq 0
              \end{align}



\section{Quadratic Assignment Problem (QAP)}
\begin{resource}{}{}
\begin{itemize}
\item \href{https://ieeexplore-ieee-org.ezproxy.lib.vt.edu/stamp/stamp.jsp?tp=&arnumber=7170278}{An applied case of quadratic assignment problem in hospital department layout}
\item See \href{https://www.semanticscholar.org/paper/Quadratic-Assignment-Problem%3A-A-survey-and-Shawky-Metwally/247b45613d6d3b6961fdad44f9e7fefb70fd3e82}{Quadratic Assignment Problem: A survey and Applications}.
\end{itemize}
\end{resource}
The quadratic assignment problem must choose the assignment of $n$ facilities to $n$ locations.  Each facility sends some flow to each other facility, and there is a distance to consider between locations.   
The objective is to minimize to distance times the flow of the assignment.


\textbf{Example:  Hospital Layout}
On any given day in the hospital, there will be patients that move from various locations in the hospital to various other locations in the hospital.  For example, patients move from the operating room to a recovery room, or from the emergency room to the operating room, etc.

We would like to chose the locations of these places in the hospital to minimize the amount of total distance traveled by all the patients. 






\begin{general}{Quadratic Assignment Problem}{\npcomplete}
Given flow $f_{ij}$  connections $c_{ij}$ and fixed building costs $f_i$, demands $d_j$ and capacities $u_i$, the capacitated facility location problem is 

\noindent \textbf{Sets:}
\begin{itemize}
\item Let $I = \{1,\dots, n\}$ be the set of facilities.
\item Let $K = \{1, \dots, n\}$ be the set of locations.
\end{itemize}

\noindent \textbf{Parameters:}
\begin{itemize}
\item $f_{ij}$ - flow from facility $i$ to facility $j$.
\item $d_{kl}$ - distance from location $k$ to location $l$.
\item $c_{ik}$ - cost to setup facility $i$ at location $k$.
\end{itemize}

\noindent \textbf{Variables:}
\begin{itemize}
\item Let 
\begin{equation*}
x_{ik} = \begin{cases}
1 & \text{if we place facility $i$ in location $k$,}\\
0 & \text{otherwise.}
\end{cases}
\end{equation*}
\end{itemize}

\noindent  \textbf{Model:}
\begin{align}
\min & \displaystyle\sum_{i=1}^n\sum_{j=1}^n \sum_{k=1}^n \sum_{l = 1}^n f_{ij}d_{kl}x_{ik}x_{jl} + \sum_{i=1}^n \sum_{k = 1}^n c_{ik} x_{ik} \tag{total cost}\\
\text{s.t.} & \displaystyle\sum_{i=1}^n x_{ik}=1 \text{ for all }k=1,\dots,n \tag{assign facility to location $k$}\\
& \displaystyle \sum_{k=1}^n x_{ik}=1 \text{ for all }i=1\dots,n \tag{assign one location to facility $i$}\\
&x_{ik}\in\{0,1\}\text{ for all } i=1,\dots,n, \text{ and } k = 1, \dots, n \tag{binary decisions}
\end{align}
\end{general}

\section{Genaralized Assignment Problem (GAP)}
\todo[inline]{Fix up this section}
\url{https://en.wikipedia.org/wiki/Generalized_assignment_problem}
In \href{applied_mathematics}{applied mathematics}, the maximum
\textbf{generalized assignment problem} is a problem in
\href{combinatorial_optimization}{combinatorial optimization}. This
problem is a \url{generalization} of the
\href{assignment_problem}{assignment problem} in which both tasks and
\href{Agent-based_model}{agents} have a size. Moreover, the size of each
task might vary from one agent to the other.

This problem in its most general form is as follows: There are a number
of agents and a number of tasks. Any agent can be assigned to perform
any task, incurring some cost and profit that may vary depending on the
agent-task assignment. Moreover, each agent has a budget and the sum of
the costs of tasks assigned to it cannot exceed this budget. It is
required to find an assignment in which all agents do not exceed their
budget and total profit of the assignment is maximized.

\hypertarget{in-special-cases}{%
\subsection{In special cases}\label{in-special-cases}}

In the special case in which all the agents' budgets and all tasks'
costs are equal to 1, this problem reduces to the
\href{assignment_problem}{assignment problem}. When the costs and
profits of all tasks do not vary between different agents, this problem
reduces to the multiple knapsack problem. If there is a single agent,
then, this problem reduces to the \href{knapsack_problem}{knapsack
problem}.

\hypertarget{explanation-of-definition}{%
\subsection{Explanation of definition}\label{explanation-of-definition}}

In the following, we have \emph{n} kinds of items, \(a_1\) through
\(a_n\) and \emph{m} kinds of bins \(b_1\) through \(b_m\). Each bin
\(b_i\) is associated with a budget \(t_i\). For a bin \(b_i\), each
item \(a_j\) has a profit \(p_{ij}\) and a weight \(w_{ij}\). A solution
is an assignment from items to bins. A feasible solution is a solution
in which for each bin \(b_i\) the total weight of assigned items is at
most \(t_i\). The solution's profit is the sum of profits for each
item-bin assignment. The goal is to find a maximum profit feasible
solution.

Mathematically the generalized assignment problem can be formulated as
an \href{Integer_programming}{integer program}:
\begin{align} \text{maximize } &
\sum_{i=1}^m\sum_{j=1}^n
p_{ij} x_{ij}. \\
\text{subject to } & \sum_{j=1}^n
w_{ij} x_{ij} \le t_i & & i=1,
\ldots, m; \\ &
\sum_{i=1}^m x_{ij} = 1 & & j=1,
\ldots, n; \\ & x_{ij}
\in \{0,1\} & & i=1,
\ldots, m, \quad j=1,
\ldots, n; \end{align}



\begin{resource}



\begin{itemize}
\item The AIMMS modeling has many great examples.  It can be book found here:\href{https://www.aimms.com/english/developers/resources/manuals/optimization-modeling}{AIMMS Modeling Book}.
\item \href{http://web.mit.edu/15.053/www/AMP-Chapter-09.pdf}{MIT Open Courseware}
\item  For many real world examples, see this book \href{https://link-springer-com.ezproxy.lib.vt.edu/book/10.1007/978-1-4939-1007-6}{
Case Studies in Operations Research
Applications of Optimal Decision Making, edited by  Murty, Katta G}.
Or find it \href{https://www.springer.com/gp/book/9781493910069}{here}.
\item \href{https://github.com/Gurobi/modeling-examples}{GUROBI modeling examples by GUROBI}
\item \href{https://github.com/open-optimization/open-optimization-or-examples/tree/master/integer-programming}{GUROBI modeling examples by Open Optimization that are linked in this book}
\end{itemize}

Knapsack Problem
\begin{itemize}
\item \href{https://www.youtube.com/watch?v=GteOoMGUOdY}{Video! - Michel Belaire (EPFL) teaching knapsack problem}
\end{itemize}

Set Cover
\begin{itemize}
\item \href{https://www.youtube.com/watch?v=cjSeHSjPmsk}{Video! - Michel Belaire (EPFL) explaining set covering problem}
\item See \href{https://download.aimms.com/aimms/download/manuals/AIMMS3OM_MediaSelection.pdf}{AIMMS - Media Selection} for an example of set covering applied to media selection.
\end{itemize}

Facility Location
\begin{itemize}
\item \href{https://en.wikipedia.org/wiki/Facility_location_problem}{Wikipedia - Facility Location Problem}
\item See \href{https://github.com/Gurobi/modeling-examples/tree/master/facility_location}{GUROBI Modeling Examples - Facility Location}.
\end{itemize}


{Other examples}
\begin{itemize}
\item \href{https://www.juliaopt.org/notebooks/JuMP-Sudoku.html}{Sudoku}
\item \href{https://download.aimms.com/aimms/download/manuals/AIMMS3OM_EmployeeTraining.pdf}{AIMMS - Employee Training}

\item \href{https://download.aimms.com/aimms/download/manuals/AIMMS3OM_MediaSelection.pdf}{AIMMS - Media Selection}

\item \href{https://download.aimms.com/aimms/download/manuals/AIMMS3OM_Diet.pdf}{AIMMS - Diet Problem}

\item \href{https://download.aimms.com/aimms/download/manuals/AIMMS3OM_FarmPlanning.pdf}{AIMMS - Farm Planning Problem}

\item \href{https://download.aimms.com/aimms/download/manuals/AIMMS3OM_Pooling.pdf}{AIMMS - Pooling Probem}

\item \href{https://www.informs.org/Impact}{INFORMS - Impact}
\item \href{https://www.informs.org/Impact/O.R.-Analytics-Success-Stories/Optimized-school-bus-routing-helps-school-districts-design-better-policies}{INFORMS - Success Story - Bus Routing}
\end{itemize}

{Notes from AIMMS modeling book.}
\begin{itemize}

\item \href{http://inside.mines.edu/~anewman/MIP_practice120212.pdf}{AIMMS - Practical guidelines for solving difficult MILPs}

\item \href{https://download.aimms.com/aimms/download/manuals/AIMMS3OM_LinearProgrammingTricks.pdf}{AIMMS - Linear Programming Tricks}


\item  \href{https://download.aimms.com/aimms/download/manuals/AIMMS3OM_FormulatingOptimizationModels.pdf}{AIMMS - Formulating Optimization Models}


\item \href{https://pdfs.semanticscholar.org/b01f/ad44c20c372fdda95cbfb980c0d37302de07.pdf}{AIMMS - Practical guidelines for solving difficult linear programs}
\end{itemize}

Modeling Tricks
\begin{itemize}
\item \href{https://jump.dev/JuMP.jl/stable/tutorials/linear/tips_and_tricks/}{JuMP tips and tricks}
\item \href{https://docs.mosek.com/modeling-cookbook/mio.html}{Mosek Modeling Cookbook}
\end{itemize}


{Further Topics}
\begin{itemize}
\item \href{https://or.stackexchange.com/questions/1319/best-model-for-precedence-constraints-within-scheduling-problem}{Precedence Constraints}
\end{itemize}
\end{resource}

%\end{document}
