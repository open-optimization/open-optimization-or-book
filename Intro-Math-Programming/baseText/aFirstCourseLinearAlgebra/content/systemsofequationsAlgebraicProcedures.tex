\section{Systems Of Equations, Algebraic Procedures}

\begin{outcome}
\begin{enumerate}
\item[A.] Use elementary operations to find the solution to a linear system of equations. 

\item[B.] Find the \ef\;and \rref\;of a matrix. 

\item[C.] Determine whether a system of linear equations has no solution, a
unique solution or an infinite number of solutions from its \ef.

\item[D.] Solve a system of equations using Gaussian Elimination and Gauss-Jordan Elimination.

\item[E.] Model a physical system with linear equations and then solve. 
\end{enumerate}

\end{outcome}

We have taken an in depth look at graphical representations of systems of equations, as well as how to find possible
solutions graphically. Our attention now turns to working with systems algebraically. 

\begin{definition}{System of Linear Equations}{systemoflinearequations}
A \textbf{system of linear equations} \index{system of equations} is a list of equations,
\begin{equation*}
\begin{array}{c}
a_{11}x_{1}+a_{12}x_{2}+\cdots +a_{1n}x_{n}=b_{1} \\
a_{21}x_{1}+a_{22}x_{2}+\cdots +a_{2n}x_{n}=b_{2} \\
\vdots \\
a_{m1}x_{1}+a_{m2}x_{2}+\cdots +a_{mn}x_{n}=b_{m}
\end{array}
\end{equation*}
where $a_{ij}$ and $b_{j}$ are real numbers. The above is a system
of $m$ equations in the $n$ variables, $x_{1},x_{2}\cdots ,x_{n}$.
 Written more simply in terms
of summation notation, the above can be written in the form
\begin{equation*}
\sum_{j=1}^{n}a_{ij}x_{j}=b_{i},
\text{ }i=1,2,3,\cdots ,m
\end{equation*}
\end{definition}

The relative size of $m$ and $n$ is not important here. Notice that we
have allowed $a_{ij}$ and $b_{j}$ to be any real number.  We can also
call these numbers \textbf{scalars} \index{scalar}. We will use this
term throughout the text, so keep in mind that the term
\textbf{scalar} just means that we are working with real numbers.

Now, suppose we have a system where $b_{i} = 0$ for all $i$. In other words every
equation equals $0$. This is a special type of system.

\begin{definition}{Homogeneous System of Equations}{homogeneoussystem}
A system of equations is called \textbf{homogeneous} \index{system of equations!homogeneous} if each equation in
the system  is equal to $0$. A homogeneous system has the form 
\begin{equation*}
\begin{array}{c}
a_{11}x_{1}+a_{12}x_{2}+\cdots +a_{1n}x_{n}= 0 \\
a_{21}x_{1}+a_{22}x_{2}+\cdots +a_{2n}x_{n}= 0  \\
\vdots \\
a_{m1}x_{1}+a_{m2}x_{2}+\cdots +a_{mn}x_{n}= 0 
\end{array}
\end{equation*}
where $a_{ij}$ are scalars and $x_{i}$ are variables.
\end{definition}

Recall from the previous section that our goal when working with systems of linear equations 
was to find the point of intersection of the equations when graphed. In other words, we 
looked for the solutions to the system. We now wish to find these solutions algebraically. We want to find values for  
$ x_{1},\cdots ,x_{n} $ which solve all of
the equations. If such a set of values exists,  we call $\left( x_{1},\cdots ,x_{n}\right)$  the \textbf{solution set}. \index{system of equations!solution set} 

Recall the above discussions about the types of solutions possible. 
We will see that systems of linear equations will have one unique solution, infinitely many solutions, 
or no solution.  Consider the following definition. 

\begin{definition}{Consistent and Inconsistent Systems}{consistentandinconsistent}
A system of linear equations is called
\index{consistent system} \textbf{consistent} if there exists at least one solution. It is
called
\index{inconsistent system} \textbf{inconsistent }if there is no solution.
\end{definition}

If you think of each equation as a condition which must
be satisfied by the variables, consistent would mean there is some choice of
variables which can satisfy \textbf{all} the conditions. Inconsistent would mean
there is no choice of the variables which can satisfy all of the conditions.

The following sections provide methods for determining if a system is consistent or inconsistent, and finding
solutions if they exist.