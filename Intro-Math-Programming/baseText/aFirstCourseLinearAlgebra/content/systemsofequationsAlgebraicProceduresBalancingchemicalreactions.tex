\subsection{Balancing Chemical Reactions}

The tools of linear algebra can also be used in the subject area of Chemistry, specifically for balancing chemical reactions. 

Consider the chemical reaction 
\index{chemical reactions!balancing} 
\begin{equation*}
SnO_{2}+H_{2}\rightarrow Sn+H_{2}O
\end{equation*}
Here the elements involved are tin ($Sn$), oxygen ($O$), and hydrogen ($H$). A
chemical reaction occurs and the result is a combination of tin ($Sn$) and water ($H_{2}O$). When considering
chemical reactions, we want to investigate how much of each element we began with and how much of each element is involved in the result.

An important theory we will use here is the mass balance theory. It tells us that we cannot create or delete 
elements within a chemical reaction. For example, in the above expression, we must have the same number of oxygen, tin,
and hydrogen on both sides of the reaction. Notice that this is not currently the case.
 For example, there are two oxygen atoms on the left and only one
on the right. In order to fix this, we want to find numbers $x,y,z,w$ such that 
\begin{equation*}
xSnO_{2}+yH_{2}\rightarrow zSn+wH_{2}O
\end{equation*}
where both sides of the reaction have the same number of atoms of the various elements. 

This is a familiar problem. We can solve it by setting up a system of equations 
in the variables $x,y,z,w$. Thus you need 
\begin{equation*}
\begin{array}{cl}
Sn: & x=z \\ 
O: & 2x=w \\ 
H: & 2y=2w
\end{array}
\end{equation*}

We can rewrite these equations as
\begin{equation*}
\begin{array}{cl}
Sn: & x - z = 0 \\ 
O: & 2x - w = 0 \\ 
H: & 2y - 2w = 0 
\end{array}
\end{equation*}

The augmented matrix for this system of equations is given by 
\begin{equation*}
\leftB 
\begin{array}{rrrr|r}
1 & 0 & -1 & 0 & 0 \\ 
2 & 0 & 0 & -1 & 0 \\ 
0 & 2 & 0 & -2 & 0
\end{array}
\rightB
\end{equation*}

The \rref \;of this matrix is 
\begin{equation*}
\leftB
\begin{array}{rrrr|r}
1 & 0 & 0 & -\vspace{0.05in}\frac{1}{2} & 0 \\ 
0 & 1 & 0 & -1 & 0 \\ 
0 & 0 & 1 & -\vspace{0.05in}\frac{1}{2} & 0
\end{array}
\rightB
\end{equation*}

The solution is given by 
\begin{equation*}
\begin{array}{c}
x - \frac{1}{2} w = 0 \\
y - w = 0 \\
z - \frac{1}{2}w = 0
\end{array}
\end{equation*} 

which we can write as 
\begin{equation*}
\begin{array}{c}
x = \frac{1}{2} t \\
y = t \\
z = \frac{1}{2}t \\
w = t
\end{array}
\end{equation*}

For example, let $w=2$ and this would yield $x=1,y=2,$ and $z=1.$ We can put these values back into 
the expression for the reaction which yields 
\begin{equation*}
SnO_{2}+2H_{2}\rightarrow Sn+2H_{2}O
\end{equation*}
Observe that each side of the expression contains the same number of atoms of each element. This means that 
it preserves the total number of atoms, as required, and so the chemical
reaction is balanced. 

Consider another example.

\begin{example}{Balancing a Chemical Reaction}{balancingchem}
Potassium is denoted by $K,$ oxygen by $O,$
phosphorus by $P$ and hydrogen by $H$. 
Consider the reaction given by
\begin{equation*}
KOH+H_{3}PO_{4}\rightarrow K_{3}PO_{4}+H_{2}O
\end{equation*}
Balance this chemical reaction.
\end{example}

\begin{solution}
We will use the same procedure as above to solve this problem. We need to find values for 
$x,y,z,w$ such that  
\begin{equation*}
xKOH+yH_{3}PO_{4}\rightarrow zK_{3}PO_{4}+wH_{2}O
\end{equation*}
preserves the total number of atoms of each element. 

Finding these values can be done by finding the solution to the following system of equations.
\begin{equation*}
\begin{array}{cl}
K: & x=3z \\ 
O: & x+4y=4z+w \\ 
H: & x+3y=2w \\ 
P: & y=z
\end{array}
\end{equation*}

The augmented matrix for this system is 
\begin{equation*}
\leftB 
\begin{array}{rrrr|r}
1 & 0 & -3 & 0 & 0 \\ 
1 & 4 & -4 & -1 & 0 \\ 
1 & 3 & 0 & -2 & 0 \\ 
0 & 1 & -1 & 0 & 0
\end{array}
\rightB
\end{equation*}
and the \rref \;is
\begin{equation*}
\leftB
\begin{array}{rrrr|r}
1 & 0 & 0 & -1 & 0 \\ 
0 & 1 & 0 & -\vspace{0.05in}\frac{1}{3} & 0 \\ 
0 & 0 & 1 & -\vspace{0.05in}\frac{1}{3} & 0 \\ 
0 & 0 & 0 & 0 & 0
\end{array}
\rightB
\end{equation*}

The solution is given by 
\begin{equation*}
\begin{array}{c}
x - w = 0 \\
y - \frac{1}{3}w = 0 \\
z - \frac{1}{3}w = 0
\end{array}
\end{equation*}
which can be written as
\begin{equation*}
\begin{array}{c}
x = t \\
y = \frac{1}{3}t \\
z = \frac{1}{3}t \\
w = t
\end{array}
\end{equation*}

Choose a value for $t$, say $3$. Then $w=3$ and this yields $x=3,y=1,z=1.$ It follows that the balanced
reaction is given by  
\begin{equation*}
3KOH+1H_{3}PO_{4}\rightarrow 1K_{3}PO_{4}+3H_{2}O
\end{equation*}
Note that this results in the same number of atoms on both sides.
\end{solution}

Of course these numbers you are finding would typically be the number of
moles of the molecules on each side. Thus three moles of $KOH$ added to one
mole of $H_{3}PO_{4}$ yields one mole of $K_{3}PO_{4}$ and three moles of 
$H_{2}O$.