\subsection{Uniqueness of the\RREF}

As we have seen in earlier sections, we know that every matrix can be brought into \rref \;by a sequence of elementary row operations. Here we will prove that the resulting matrix is unique; in other words, the resulting matrix in \rref\;does not depend upon the particular sequence of elementary row operations or the order in which they were performed. 

Let $A$ be the augmented matrix of a homogeneous system of linear
equations in the variables $x_1, x_2, \cdots, x_n$ which is also in
\rref. The matrix $A$ divides the set of variables in two different
types. We say that $x_i$ is a {\em basic variable}  \index{variable!basic}   whenever $A$ has
a leading $1$ in column number $i$, in other words, when column $i$ is
a pivot column. Otherwise we say that $x_i$ is a {\em free
variable\em}. \index{variable!free}

Recall Example \ref{exa:twoparametersetofsoln}.

\begin{example}{Basic and Free Variables}{basicfree}
Find the basic and free variables in the system
\[
\begin{array}{c}
x+2y-z+w=3 \\
x+y-z+w=1 \\
x+3y-z+w=5
\end{array}
\]
\end{example}

\begin{solution}
Recall from the solution of Example \ref{exa:twoparametersetofsoln} that the \ef \;of the augmented matrix of this system is given by
\[
\leftB
\begin{array}{rrrr|r}
1 & 2 & -1 & 1 & 3 \\
0 & 1 & 0 & 0 & 2 \\
0 & 0 & 0 & 0 & 0
\end{array}
\rightB  
\]
You can see that columns $1$ and $2$ are pivot columns. These columns correspond to variables $x$ and $y$, making these the basic variables. Columns $3$ and $4$ are not pivot columns, which means that $z$ and $w$ are free variables.

We can write the solution to this system as 
\[
\begin{array}{c}
x=-1+s-t \\
y=2 \\
z=s \\
w=t
\end{array}
\]

Here the free variables are written as parameters, and the basic variables are given by linear functions of these parameters. 
\end{solution}

In general, all solutions can be written in terms of the free variables. In such a description, the free variables can take any values (they become parameters), while the basic variables become simple linear functions of these parameters. Indeed, a basic variable $x_i$ is a linear function of {\em only \em}those free variables $x_j$ with $j>i$. This leads to the following observation.

\begin{proposition}{Basic and Free Variables}{basicfree}
If $x_i$ is a basic variable of a homogeneous system of linear equations, then any solution of the system with $x_j=0$ for all those free variables $x_j$ with $j>i$ must also have $x_i=0$.
\end{proposition}

Using this proposition, we prove a lemma which will be used in the proof of the main result of this section below.

\begin{lemma}{Solutions and the \RREF \;of a Matrix}{rrefsolutions}
Let $A$ and $B$ be two distinct augmented matrices for two homogeneous systems of $m$ equations in $n$ variables, such that $A$ and $B$ are each in \rref. Then, the two systems do not have exactly the same solutions.
\end{lemma}
\ifdefined\showproofs
\begin{proof}
With respect to the linear systems associated with the matrices $A$ and $B$, there are two cases to consider:
\begin{itemize}
\item Case $1$: the two systems have the same basic variables
\item Case $2$: the two systems do not have the same basic variables
\end{itemize}
In case $1$, the two matrices will have exactly the same pivot positions. However, since $A$ and $B$ are not identical, there is some row of $A$ which is different from the corresponding row of $B$ and yet the rows each have a pivot in the same column position. Let $i$ be the index of this column position. Since the matrices are in \rref, the two rows must differ at some entry in a column $j>i$. Let these entries be $a$ in $A$ and $b$ in $B$, where $a \neq b$. Since $A$ is in \rref, if $x_j$ were a basic variable for its linear system, we would have $a=0$. Similarly, if $x_j$ were a basic variable for the linear system of the matrix $B$, we would have $b=0$. Since $a$ and $b$ are unequal, they cannot both be equal to $0$, and hence $x_j$ cannot be a basic variable for both linear systems. However, since the systems have the same basic variables, $x_j$ must then be a free variable for each system. We now look at the solutions of the systems in which $x_j$ is set equal to $1$ and all other free variables are set equal to $0$. For this choice of parameters, the solution of the system for matrix $A$ has $x_j=-a$, while the solution of the system for matrix $B$ has $x_j=-b$, so that the two systems have different solutions.

In case $2$, there is a variable $x_i$ which is a basic variable for one matrix, let's say $A$, and a free variable for the other matrix $B$. The system for matrix $B$ has a solution in which $x_i=1$ and $x_j=0$ for all other free variables $x_j$. However, by Proposition \ref{prop:basicfree} this cannot be a solution of the system for the matrix $A$. This completes the proof of case $2$.
\end{proof}
\fi
Now, we say that the matrix $B$ is \textbf{equivalent} \index{matrix!equivalent} to the matrix $A$ provided that $B$ can be obtained from $A$ by performing a sequence of elementary row operations beginning with $A$. The importance of this concept lies in the following result.

\begin{theorem}{Equivalent Matrices}{equivalent}
The two linear systems of equations corresponding to two equivalent augmented matrices have exactly the same solutions.
\end{theorem}

The proof of this theorem is left as an exercise. 

Now, we can use Lemma \ref{lem:rrefsolutions} and Theorem \ref{thm:equivalent} to prove the main result of this section.

\begin{theorem}{Uniqueness of the \RREF}{uniquerref}
Every matrix $A$ is equivalent to a unique matrix in \rref.
\end{theorem}

\ifdefined\showproofs
\begin{proof}
Let $A$ be an $m \times n$ matrix and let $B$ and $C$ be matrices in \rref, each equivalent to $A$. It suffices to show that $B=C$.

Let $A^{+}$ be the matrix $A$ augmented with a new rightmost column consisting entirely of zeros. Similarly, augment matrices $B$ and $C$ each with a rightmost column of zeros to obtain $B^{+}$ and $C^{+}$. Note that $B^{+}$ and $C^{+}$
are matrices in \rref \;which are obtained from $A^{+}$ by respectively applying the same sequence of elementary row operations which were used to obtain $B$ and $C$ from $A$. 

Now, $A^{+}$, $B^{+}$, and $C^{+}$ can all be considered as augmented matrices of homogeneous linear systems in the variables $x_1, x_2, \cdots, x_n$. Because $B^{+}$ and $C^{+}$ are each equivalent to $A^{+}$, Theorem \ref{thm:equivalent} ensures that all three homogeneous linear systems have exactly the same solutions. By Lemma \ref{lem:rrefsolutions} we conclude that $B^{+}=C^{+}$. By construction, we must also have $B=C$. 
\end{proof}
\fi
According to this theorem we can say that each matrix $A$ has a unique \rref.