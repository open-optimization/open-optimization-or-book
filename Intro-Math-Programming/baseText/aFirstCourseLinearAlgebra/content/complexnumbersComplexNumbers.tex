\section{Complex Numbers}

\begin{outcome}
\begin{enumerate}
\item[A.]  Understand the geometric significance of a complex number as a point
in the plane.

\item[B.]  Prove algebraic properties of addition and multiplication of complex numbers, and 
apply these properties. Understand the action of taking the conjugate of a complex number.

\item[C.]  Understand the absolute value of a complex number and how to find it
as well as its geometric significance.
\end{enumerate}
\end{outcome}

Although very powerful, the real numbers are inadequate to solve
equations such as $x^2+1=0$, and this is where complex numbers come
in. We define the number $i$ as the imaginary number such that $i^2 =
-1$, and define complex numbers as those of the form $z = a + bi$
where $a$ and $b$ are real numbers. We call this the standard form, or Cartesian form, of the complex number
$z$.
\index{complex numbers!standard form} Then, we refer to $a$ as the
{\em real\em} part of $z$, and $b$ as the {\em imaginary\em} part of
$z$. It turns out that such numbers not only solve the above equation,
but in fact also solve any polynomial of degree at least 1 with complex coefficients. This property, called the Fundamental Theorem of Algebra, 
\index{Fundamental Theorem of Algebra} is sometimes referred to by saying $\mathbb{C}$ is
algebraically closed. Gauss is usually credited with giving a proof
of this theorem in 1797 but many others worked on it and the first
completely correct proof was due to Argand in 1806.

Just as a real number can be considered as a point on the line, a
complex number $z = a + bi$ can be considered as a point $\left(
a,b\right) $ in the plane whose $x$ coordinate is $ a$ and whose $y$
coordinate is $b.$ For example, in the following picture, the point $z
= 3+2i$ can be represented as the point in the plane with
coordinates $\left( 3,2\right) .$

\begin{center}
\begin{tikzpicture}
\draw[<->](3.5,0)--(0,0)--(0,2.5);
\draw(-0.2,2)--(0.2,2);
\draw(-0.2,1)--(0.2,1);
\draw(1,0.2)--(1,-0.2);
\draw(2,0.2)--(2,-0.2);
\draw(3,0.2)--(3,-0.2);
\draw[help lines, dotted] (0,2)--(3,2)--(3,0);
\draw[fill, red](3,2) circle [radius=1pt];
\node[right] at (3,2){$z = (3,2)= 3+2i$};
\end{tikzpicture}
\end{center}

Addition of complex numbers is defined as follows. \index{complex numbers!addition}
\begin{equation*}
\left( a+bi\right) +\left( c+di\right) =\left( a+c\right) +\left( b+d\right)i
\end{equation*}

This addition obeys all the usual properties \index{field axioms} as the following theorem indicates.

\begin{theorem}{Properties of Addition of Complex Numbers}{propertiesofadditioncomplexnumbers}
Let $z,w,$ and $v$ be complex numbers. Then the following properties hold.
\begin{itemize}
\item Commutative Law for Addition
\begin{equation*}
z+w=w+z
\end{equation*}

\item Additive Identity
\begin{equation*}
z+0=z
\end{equation*}

\item Existence of Additive Inverse
\begin{equation*}
\begin{array}{l}
\mbox{For each} \; z\in \mathbb{C}, \mbox{there exists}\; -z\in \mathbb{C} \mbox{ such that}\; 
z+\left( -z\right) =0 \\
\mbox{In fact if } z=a+bi, \mbox{ then } -z=-a-bi.
\end{array}
\end{equation*}

\item Associative Law for Addition
\begin{equation*}
\left( z+w\right) +v= z +\left( w+v\right)
\end{equation*}
\end{itemize}
\end{theorem}

The proof of this theorem is left as an exercise for the reader.

Now, multiplication of complex numbers is defined the way you would expect, recalling that $i^{2} = -1$.
\index{complex numbers!multiplication}
\begin{eqnarray*}
\left( a+bi\right) \left( c+di\right) &=&ac+adi+bci+i^{2}bd \\
&=&\left( ac-bd\right) +\left( ad + bc \right)i 
\end{eqnarray*}

Consider the following examples.

\begin{example}{Multiplication of Complex Numbers}{complexmultiplication}
\begin{itemize}
\item $(2-3i)(-3+4i) = 6+17i$
\item $(4-7i)(6-2i) = 10-50i$
\item $(-3+6i)(5-i) = -9+33i$
\end{itemize}
\end{example}

The following are important properties of multiplication of complex numbers.

\begin{theorem}{Properties of Multiplication of Complex Numbers}{propertiesmultiplicationcomplexnumbers}
Let $z,w$ and $v$ be complex numbers. Then, the following properties of multiplication hold.

\begin{itemize}

\item Commutative Law for Multiplication
\begin{equation*}
zw=wz
\end{equation*}

\item Associative Law for Multiplication
\begin{equation*}
\left( zw\right) v=z\left( wv\right) 
\end{equation*}

\item Multiplicative Identity
\begin{equation*}
1z=z
\end{equation*}

\item Existence of Multiplicative Inverse
\begin{equation*}
\mbox{For each}\; z\neq 0, \mbox{there exists}\; z^{-1} \mbox{ such that}\; zz^{-1}=1
\end{equation*}

\item Distributive Law
\begin{equation*}
z\left( w+v\right) =zw+zv
\end{equation*}
\end{itemize}
\end{theorem}

You may wish to verify some of these statements.  The real numbers
also satisfy the above axioms, and in general any mathematical
structure which satisfies these axioms is called a field. There are
many other fields, in particular even finite ones particularly useful
for cryptography, and the reason for specifying these axioms is that
linear algebra is all about fields and we can do just about anything
in this subject using any field. Although here, the fields of most
interest will be the familiar field of real numbers, denoted as
$\mathbb{R}$, and the field of complex numbers, denoted as
$\mathbb{C}$.

An important construction regarding complex numbers is the complex
conjugate
\index{complex numbers!conjugate}denoted by a horizontal line above the number, $\overline{z}$. It
is defined as follows.

\begin{definition}{Conjugate of a Complex Number}{conjugate}
Let $z = a+bi$ be a complex number. Then the \textbf{conjugate} of $z$, written $\overline{z}$ is given by 
\begin{equation*}
\overline{a+bi}= a-bi
\end{equation*}
\end{definition}

Geometrically, the action of the conjugate is to reflect a given complex number across the $x$ axis.
Algebraically, it changes the sign on the imaginary part of the complex number. Therefore, for a real number $a$, $\overline{a} = a$. 

\begin{example}{Conjugate of a Complex Number}{complexconjugate}
\begin{itemize}

\item If $z=3+4i$, then $\overline{z}=3-4i$,
i.e., $\overline{3+4i}=3-4i$.

\item $\overline{-2+5i}= -2-5i$.

\item $\overline{i}= -i$.

\item $\overline{7}= 7$.
\end{itemize}
\end{example}

Consider the following computation. 
\begin{eqnarray*}
\left( \overline{a+bi}\right) \left( a+bi\right) &=&\left( a-bi\right)
\left( a+bi\right) \\
&=&a^{2}+b^{2}-\left( ab-ab\right)i =a^{2}+b^{2}
\end{eqnarray*}
Notice that there is no imaginary part in the product, thus
multiplying a complex number by its conjugate results in  a real number.

\begin{theorem}{Properties of the Conjugate}{propertiesconjugate}
Let $z$ and $w$ be complex numbers. Then, the following properties of the conjugate hold.

\begin{itemize}
\item
$\overline{z\pm w} = \overline{z} \pm \overline{w}$.
\item
$\overline{(zw)} = \overline{z}~ \overline{w}$.
\item
$\overline{(\overline{z})}=z$.
\item
$\overline{\left(\frac{z}{w}\right)} =
\frac{\overline{z}}{\overline{w}}$.
\item
$z$ is real if and only if $\overline{z}=z$.
\end{itemize}
\end{theorem}

Division of complex numbers is defined as follows. Let $z=a+bi$ and $w=c+di$ be complex numbers such that $c,d$ are not both zero. Then the quotient $z$ divided by $w$ is
\begin{eqnarray*}
\frac{z}{w}=
\frac{a+bi}{c+di} & = & \frac{a+bi}{c+di}\times \frac{c-di}{c-di} \\
& = & \frac{(ac+bd)+(bc-ad)i}{c^2+d^2} \\
& = & \frac{ac+bd}{c^2+d^2} +\frac{bc-ad}{c^2+d^2}i.
\end{eqnarray*}

In other words, the quotient $\frac{z}{w}$ is obtained by multiplying
both top and bottom of $\frac{z}{w}$ by $\overline{w}$ and
then simplifying the expression.

\begin{example}{Division of Complex Numbers}{divisioncomplex}
\begin{itemize}
\item
\[ \frac{1}{i} = \frac{1}{i}\times \frac{-i}{-i}
=\frac{-i}{-i^2}=-i \]

\item
\[ \frac{2-i}{3+4i} = \frac{2-i}{3+4i}\times \frac{3-4i}{3-4i}
=\frac{(6-4)+(-3-8)i}{3^2+4^2}
=\frac{2-11i}{25}
=\frac{2}{25} - \frac{11}{25}i \]

\item

\[ \frac{1-2i}{-2+5i} = \frac{1-2i}{-2+5i}\times \frac{-2-5i}{-2-5i}
=\frac{(-2-10) + (4-5)i}{2^2+5^2}
=-\frac{12}{29}-\frac{1}{29}i  \]
\end{itemize}
\end{example}

Interestingly every nonzero complex number $a+bi$ has a unique
multiplicative inverse. In other words, for a nonzero complex number
$z$, there exists a number $z^{-1}$ (or $\frac{1}{z}$) so that
$zz^{-1} = 1$. Note that $z=a+bi$ is nonzero exactly when
$a^{2}+b^{2}\neq 0$, and its inverse can be written in standard form as defined now. 

\begin{definition}{Inverse of a Complex Number}{complexinverse}
Let $z = a+bi$ be a complex number. Then the multiplicative inverse of $z$, written $z^{-1}$ exists if and only if $a^{2}+b^{2}\neq 0$ and is given by 
\begin{equation*}
z^{-1} = \frac{1}{a+bi}  = \frac{1}{a+bi}\times \frac{a-bi}{a-bi}=\frac{a-bi}{a^{2}+b^{2}}=\frac{a}{a^{2}+b^{2}}-i\frac{b}{
a^{2}+b^{2}}
\end{equation*}
\end{definition}

Note that we may write $z^{-1}$ as $\frac{1}{z}$. Both notations represent the multiplicative inverse of the complex number $z$. Consider now an example.

\begin{example}{Inverse of a Complex Number}{complexinverse}
Consider the complex number $z = 2 + 6i$. Then $z^{-1}$ is defined, and
\begin{eqnarray*}
 \frac{1}{z} &=& \frac{1}{2+6i}\\
 &=& \frac{1}{2+6i}\times \frac{2-6i}{2-6i} \\
 &=& \frac{2-6i}{2^2+6^2} \\
 &=& \frac{2-6i}{40} \\
 &=& \frac{1}{20} -  \frac{3}{20}i
\end{eqnarray*}

You can always check your answer by computing $zz^{-1}$.
\end{example}

Another important construction of complex numbers is that of the absolute value, also called the modulus. Consider the following definition. 

\begin{definition}{Absolute Value}{complexabsvalue}
\index{complex numbers!modulus}\index{complex numbers!absolute value} 
The absolute
value, or modulus, of a complex number, denoted $\left| z \right|$ is defined as follows.
\begin{equation*}
\left| a+bi\right| =
\sqrt{a^{2}+b^{2}}
\end{equation*}
\end{definition}

Thus, if $z$ is the complex number $z=a+bi$, it follows that
\begin{equation*}
\left| z\right| =\left( z\overline{z}\right) ^{1/2}
\end{equation*}

Also from the definition, if $z=a+bi$ and $w=c+di$ are two complex numbers,
then $\left\vert zw\right\vert =\left\vert z\right\vert \left\vert
w\right\vert .$ Take a moment to verify this.

The triangle inequality
\index{complex numbers!triangle inequality} is an important property of the absolute value
of complex numbers. There are two useful versions which we present
here, although the first one is officially called the triangle inequality. 

\begin{proposition}{Triangle Inequality}{triangleinequalitycomplex}
Let $z,w$ be complex numbers. 

The following two inequalities hold for any  complex numbers $z,w$:
\begin{equation*}
\begin{array}{l}
\left| z+w\right| \leq \left| z\right| +\left| w\right|  \\
\left| \left| z\right| -\left| w\right| \right| \leq \left| z-w\right| 
\end{array}
\end{equation*}
The first one is called the \em{Triangle Inequality}.
\end{proposition}

\begin{proof}
Let $z=a+bi$ and $w=c+di$. First note that
\begin{equation*}
z
\overline{w}=\left( a+bi\right) \left( c-di\right) =ac+bd+\left(
bc-ad\right)i
\end{equation*}
and so $\left\vert ac+bd\right\vert \leq \left\vert z\overline{w}\right\vert
=\left\vert z\right\vert \left\vert w\right\vert .$

Then,
\begin{equation*}
\left\vert z+w\right\vert ^{2}=\left( a+c+i\left( b+d\right) \right) \left(
a+c-i\left( b+d\right) \right)
\end{equation*}
\begin{equation*}
=\left( a+c\right) ^{2}+\left( b+d\right)
^{2}=a^{2}+c^{2}+2ac+2bd+b^{2}+d^{2}
\end{equation*}
\begin{equation*}
\leq \left\vert z\right\vert ^{2}+\left\vert w\right\vert ^{2}+2\left\vert
z\right\vert \left\vert w\right\vert =\left( \left\vert z\right\vert
+\left\vert w\right\vert \right) ^{2}
\end{equation*}

Taking the square root, we have that 
\begin{equation*}
\left\vert z+w\right\vert \leq \left\vert z\right\vert
+\left\vert w\right\vert 
\end{equation*}
so this verifies the triangle inequality. 

To get the  second inequality, write
\begin{equation*}
z=z-w+w,\;w=w-z+z
\end{equation*}
and so by the first form of the inequality we get both:
\begin{equation*}
\left\vert z\right\vert \leq \left\vert z-w\right\vert +\left\vert
w\right\vert ,\;\left\vert w\right\vert \leq \left\vert z-w\right\vert
+\left\vert z\right\vert
\end{equation*}

Hence,  both $\left\vert z\right\vert -\left\vert w\right\vert $ and 
$\left\vert w\right\vert -\left\vert z\right\vert $ are no larger than 
$\left\vert z-w\right\vert $. This proves the second version because 
$\left\vert \left\vert z\right\vert -\left\vert w\right\vert \right\vert $ is
one of $\left\vert z\right\vert -\left\vert w\right\vert $ or $\left\vert
w\right\vert -\left\vert z\right\vert $. 
\end{proof}

With this definition, it is important to note the following. You may wish to take
the time to verify this remark.

Let $z=a+bi$ and $w=c+di.$ Then $\left| z-w\right| =\sqrt{\left(
a-c\right) ^{2}+\left( b-d\right) ^{2}}.$ Thus the distance between the
point in the plane determined by the ordered pair $\left( a,b\right) $ and
the ordered pair $\left( c,d\right) $ equals $\left| z-w\right| $ where $z$
and $w$ are as just described.

For example, consider the distance between $\left( 2,5\right) $ and $\left(
1,8\right) .$ Letting $z=2+5i$ and $w=1+8i,$ $z-w=1-3i$, $\left( z-w\right)
\left( \overline{z-w}\right) =\left( 1-3i\right) \left( 1+3i\right)
= 10$ so $\left\vert z-w\right\vert =\sqrt{10}$.

Recall that we refer to $z=a+bi$ as the standard form of the complex number. In the next section, 
we examine another form in which we can express the complex number. 