%By Kevin Cheung
%The book is licensed under the
%\href{http://creativecommons.org/licenses/by-sa/4.0/}{Creative Commons
%Attribution-ShareAlike 4.0 International License}.
%
%This file has been modified by Robert Hildebrand 2020.  
%CC BY SA 4.0 licence still applies.

\section{Farkas' Lemma}\label{farkas-lemma}

A well-known result in linear algebra states that a system of linear
equations \(\mm{A}\vec{x} = \vec{b}\), where
\(\mm{A} \in \R^{m\times n},\) \(\vec{b}\in \R^m,\) and
\(\vec{x} = \begin{bmatrix} x_1\\ \vdots \\ x_n\end{bmatrix}\) is a
tuple of variables, has no solution if and only if there exists
\(\vec{y} \in \R^m\) such that \(\vec{y}^\T\mm{A} = \vec{0}\) and
\(\vec{y}^\T \vec{b} \neq 0\).

It is easily seen that if such a \(\vec{y}\) exists, then the system
\(\mm{A}\vec{x} = \vec{b}\) cannot have a solution. (Simply multiply
both sides of \(\mm{A}\vec{x} = \vec{b}\) on the left by
\(\vec{y}^\T\).) However, proving the converse requires a bit of work. A
standard elementary proof involves using Gauss-Jordan elimination to
reduce the original system to an equivalent system
\(\mm{Q}\vec{x} = \vec{d}\) such that \(\mm{Q}\) has a row of zero, say
in row \(i\), with \(\vec{d}_i \neq 0\). The process can be captured by
a square matrix \(\mm{M}\) satisfying \(\mm{M}\mm{A} = \mm{Q}\). We can
then take \(\vec{y}^\T\) to be the \(i\)th row of \(\mm{M}\).

An analogous result holds for systems of linear inequalities. The
following result is one of the many variants of a result known as the
\textbf{Farkas' Lemma}:

\begin{theorem}{Farkas' Lemma}{}
\protect\hypertarget{thm:farkas}{}{\label{thm:farkas}}With \(\mm{A}\),
\(\vec{x}\), and \(\vec{b}\) as above, the system
\(\mm{A}\vec{x} \geq \vec{b}\) has no solution if and only if there
exists \(\vec{y} \in \R^m\) such that
\[\vec{y} \geq \vec{0},~\vec{y}^\T \mm{A} = \vec{0},~
\vec{y}^\T\vec{b} \gt 0.\]
\end{theorem}

In other words, the system \(\mm{A}\vec{x} \geq \vec{b}\) has no
solution if and only if one can infer the inequality \(0 \geq \gamma\)
for some \(\gamma \gt 0\) by taking a nonnegative linear combination of
the inequalities.

This result essentially says that there is always a certificate (the
\(m\)-tuple \(\vec{y}\) with the prescribed properties) for the
infeasibility of the system \(\mm{A}\vec{x} \geq \vec{b}\). This allows
third parties to verify the claim of infeasibility without having to
solve the system from scratch.

\begin{example}{}{}
\protect\hypertarget{ex:unnamed-chunk-2}{}{\label{ex:unnamed-chunk-2}} For
the system
\begin{align*}
2x - y + z & \geq 2 \\
-x + y - z & \geq 0 \\
   - y + z & \geq 0,
\end{align*}

adding two times the second inequality and the third inequality to the
first inequality gives \(0 \geq 2\). Hence,
\(\vec{y} = \begin{bmatrix} 1\\ 2 \\ 1\end{bmatrix}\) is a certificate
of infeasibility for this example.
\end{example}

We now give a proof of Theorem \ref{thm:farkas}. It is easy to see that
if such a \(\vec{y}\) exists, then the system
\(\mm{A}\vec{x} \geq \vec{b}\) has no solution.
