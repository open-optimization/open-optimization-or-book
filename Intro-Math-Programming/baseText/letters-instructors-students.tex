\section*{Letter to instructors}
\todo[inline]{Fill in letter to professors and how to use the book}
This is an introductory book for students to learn optimization theory, tools, and applications.  The two main goals are (1) students are able to apply the of optimization in their future work or research (2) students understand the concepts underlying optimization solver and use this knowledge to use solvers effectively.

This book was based on a sequence of course at Virginia Tech in the Industrial and Systems Engineering Department.   The courses are \emph{Deterministic Operations Research I} and \emph{Deterministic Operations Reserach II}.   The first course focuses on linear programming, while the second course covers integer programming an nonlinear programming.  As such, the content in this book is meant to cover 2 or more full courses in optimization.

The book is designed to be read in a linear fashion. That said, many of the chapters are mostly independent and could be rearrranged, removed, or shortened as desited.  This is an open source textbook, so use it as you like.  You are encouraged to share adaptations and improvements so that this work can evolve over time.




\section*{Letter to students}



This book is designed to be a resource for students interested in learning what optimizaiton is, how it works, and how you can apply it in your future career.  The main focus is being able to apply the techniques of optimization to problems using computer technology while understanding (at least at a high level) what the computer is doing and what you can claim about the output from the computer.

Quite importantly, keep in mind that when someone claims to have \emph{optimized} a problem, we want to know what kind of gaurantees they have about how good their solution is.  Far too often, the solution that is provided is suboptimal by 10\%, 20\%, or even more.  This can mean spending excess amounts of money, time, or energy that could have been saved. And when problems are at a large scale, this can easily result in millions of dollars in savings.

For this reason, we will learn the perspective of \emph{mathematical programming} (a.k.a. mathematical optimization).  The key to this study is that we provide gaurantees on how good a solution is to a given problem.  We will also study how difficult a problem is to solve.  This will help us know (a) how long it might take to solve it and (b) how good a of a solution we might expect to be able to find in reasonable amount of time. We will later study \emph{heuristic} methods - these methods typically do not come with gaurantees, but tend to help find quality solutions.

Note: Although there is some computer programming required in this work, this is not a course on programming.   Thanks to the fantastic modelling packages available these days, we are able to solve complicated problems with little programming effort.   The key skill we will need to learn is \emph{mathematical modeling}: converting words and ideas into numbers and variables in order to communicate problems to a compter so that it can solve a problem for you.  

As a main element of this book, we would like to make the process of using code and software as easy as possible.  Attached to most examples in the book, there will be links to code that implements and solves the problem using several different tools from Excel and Python.  These examples can should make it easy to solve a similar problem with different data, or more generally, can serve as a basis for solving related problems with similar structure.